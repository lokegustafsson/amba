Eftersom vår projektgrupp har arbetat med en taskbaserad arbetsmetod har vi
inte utdelat specifika roller. Därför kommer jag lista, översiktligt, vad de
task jag har arbetat med har inneburit.

Även till rapportskrivningen har vi samarbetat och omarbetat varandras texter
och det inte är en enda person som står ensam för text i rapporten i stora
drag. Men för att exemplifiera har jag omarbetat om skrivit om
inledningsavsnittet. Det fanns också text till inledningen skriven innan som
jag har skrivit om i nästan sin helhet till slutrapporten. Det finns också
delar som Linus och Albin hade skrivit som jag har omarbetat såsom början och
syfte. Vidare har jag skrivit nya delavsnitt och omarbetat teoriavsnittet för
att bli mer sammanhållet. Avsnitt som jag i stora drag har skrivit
självständigt är avsnitt 2.2 (Motivering till symbolisk exekvering) och 2.5
(Exekveringsmotor). Det har tagits fram genom att skriva nytt men också
omarbeta gammal text skrivna av andra. Jag har också skrivit några andra delar,
exempelvis avsnitt 5.2 till 5.2.1. (S2E och AmbaPlugin).

När det gäller den tekniska biten av projektet har även det varit taskbaserad.
De task jag har utfört har inneburit exempelvis funktionalitet för att gräva
fram källkodsrader motsvarande en intervall av binär maskinkod. Detta används i
projektet för att visa en rikare dissambly i noder i grafvyn och förenklar
debugging under utvecklingen. Vidare har jag läst relativt stora mängder
källkod till det verktyg (S2E) vi har baserat vårt verktyg på. Detta för att ta
reda på hur andra plugin har utfört olika saker och de funktionaliteter i S2E
som vi kan utnyttja i vårt projekt. Detta har resulterat t.ex. i det jag gjort
för att filtrera bort all kernelkod och userspace kod som körs under analysen.
Eftersom S2E tillåter undersökning av all kernelspace och userspace kod när man
kör en analys skulle en control flow graf eller andra graf bli väldigt stora om
vi tillät inspektion av all kod som körs från uppstart till nedstängning. Detta
är alltså onödigt och vi bryr oss endast om kod som körs under tiden vårt
program som analyseras är igång och körs i virtuella maskinen (QEMU). Jag har
vidare gjort små tasks för att t.ex. undersöka och införa ett sätt att använda
QMP för att kommunicera till QEMU instansen men detta visade sig vara onödigt
och vi hade ingen användning av att prata med QEMU via QMP då det bara styr
interna saker relaterade till QEMU och inte S2E som kör inuti den. Jag har
också undersökt en del hur vi kan utvinna symboliska uttryck i en textuell
representerbar form men detta medförde inte mycket resultat.

Vidare har jag varit sekreterare och tagit mötesanteckningar under de flesta
gruppmöten vi har haft och tagit på mig att boka grupprum till våra möten
(förutom mot slutet då). Jag har periodvis varit mindre aktiv men jag är
överlag nöjd med mina bidrag.

