% beskriv vad du har jobbat med
% vilka roller du hafft
% vad du har skrivit på rapporten
% Vilka projekt roller, del uppgifter och vilka delar av slutrapporten som studenten i fråga har skrivit.

Under projektets gång har jag arbetat mycket med att undersöka relaterade akademiska fält och relaterade arbeten inom området för binäranalys. 
Detta innefattade att studera tidigare forskning och arbeten för att få en djupare förståelse för ämnet. Genom att förvärva relevant information 
och analysera tidigare arbete har jag kunnat utveckla en stark grund för projektets inriktning. Jag har varit med och utformat arbetets syfte och 
deltagit aktivt i planeringen av projektet, att definiera mål och syfte, bestämma arbetsfördelning, upprätta tidslinjer och sätta upp milstolpar. 
Bidra till projektplaneringen, säkerställa att projektet är välstrukturerat och har en klar riktning. Samlat in relevant information, studera 
tidigare arbeten och undersöka aktuella tekniska lösningar. Analyserat forskningsresultaten och bidragit till att forma projektets inriktning och 
identifiera bästa metoder. Deltagit aktivt i gruppdiskussioner och möten, och bidragit till att skapa en samarbetsvänlig atmosfär. Delat idéer, 
gett och tagit emot feedback och arbetat tillsammans med de andra medlemmarna för att lösa problem och övervinna utmaningar. Bidragit till gruppens 
sammanhållning och produktivitet. Detta har varit värdefullt för att säkerställa att våra arbetsmetoder och implementeringar är så effektiva och 
framgångsrika som möjligt. En av mina viktiga bidrag har varit att aktivt delta i planeringsfasen av projektet. Tillsammans med resten av gruppen 
har jag hjälpt till att formulera projektets syfte och mål. Vi har också arbetat tillsammans för att fördela arbetsuppgifterna på ett effektivt sätt 
och skapa en tydlig tidsplan med olika milstolpar. Genom att bidra till projektplaneringen har vi säkerställt att projektet är 
strukturerat och har en tydlig riktning att följa.

När det gäller rapporten har jag i dess nuvarande form har jag bidragit till "Abstract" samt skrivit avsnitten "2.3 Symbolisk exekvering", 
"3.2 Dynamiska binäranalysramverk för symbolisk exekvering" och "3.3 Automatiska fuzzers". Genom mitt skrivande har jag strävat efter att 
kommunicera våra idéer. I tidigare versioner av rapporten har jag också skrivit avsnitt som diskuterar projektets begränsningar och ger 
en översikt över tidigare arbeten inom området. 

I mitt bidrag till projektet har jag arbetat med flera förändringar och anpassningar för att förbättra graf 
representationen och datastrukturen för komprimerade noder. En av de viktigaste förändringarna är att omforma 
komprimerade noder till en ordnad lista, vilket skapar en mer strukturerad och lättillgänglig representation av dessa noder. 
Dessutom har jag anpassat komprimeringsalgoritmen för att vid introduktionen av nya kanter endast separera de noder 
i en komprimerad nod som inte längre kan ingå. Detta minskar onödiga åtgärder och effektiviserar hanteringen av 
komprimerade noder.

Utöver dessa ändringar har jag även arbetat med att uppdatera alla funktioner som involverar grafer där den komprimerade grafens 
implementation spelar en roll. Detta innebär att anpassa och optimera funktionerna för att säkerställa att de är kompatibla med 
de nya förbättringarna i graf representationen och datastrukturen för komprimerade noder. Dessutom har jag granskat funktioner där 
förändringar kan påverka nodernas implementation och genomfört de nödvändiga uppdateringarna för att bibehålla en enhetlig och 
korrekt fungerande implementering.

