Jag har spenderat en överväldigande majoritet av min tid med koden
istället för texten. I projektplanen skrev jag på avsnitt 3 (Problem)
och 4 (Avgränsningar) tillsammans med Clara, primärt avsnitt 3. Sen
blev i princip all text jag skrev reviderad och omskriven av andra i
den slutgiltiga inlämnade versionen.

I början av implementationen av projektet har jag undersökt
möjligheterna att skriva så lite C++ som möjligt, med bindgen, cxx och
till slut autocxx.  Striden med autocxx tog mycket tid i början och en
större bugg i projektet upptäcktes och rapporterades uppstream (och
fixades rätt fort). Till slut bestämde jag dock att det var för mycket
och övertalade gruppen istället att acceptera en mer polyglot
situation.

Sen gjorde jag också tidig research av rimliga gui-situationer, om det
var vettigt att hålla saker inom redan avnända språk i projeket (=
Rust) eller om det var vettigare att använda en webbfrontend. Kom till
slutsatsen att GTK4 hade varit rimligt nog och lämnade sen över frågan
till Clara och Albin (som sen valde att använda egui istället på grund
av ett nodrenderingsbibliotek de hittade men inte använde till slut).

Sen var jag den första som började skriva kod på AMBA. Under
sportlovet var jag isolerad från resten av gruppen och använde S2E för
att skriva en Valgrindklon som spårade allokationer och om
minnesaccesser var inom dessa allokationer eller stacken (ja, jag
glömde bss). Dock så var inte Lokes bråk med S2E lösta vid det här
laget och inget var körbart utan var mer en utforskning av S2Es
pluginapi. Koden fungerade inte och komponenten användes inte till slut.

Sen skrev jag en större komponent för att bygga, optimera/minimera och
spåra riktade grafer, som vi sen kopplade in i S2E och utgjorde
grunden till vad som sen transformeras till grafen som visas.  Hade
stora strider med prestandan och lyckades optimera en grafkomprimering
från 11 minuter ner till 50ms, något som sen förbättrades ytterligare
genom att göra byggandet av den optimerade grafen inkrementellt med
varje ny kant som lades till.

Sen jobbade jag på att beräkna strongly connected components, vilket
används för färgläggningen av noder.

Integrerade också grafbyggandet in i AmbaPlugin.

Närmare slutet jobbade jag på att få in prioriteringen av states i
AmbaPlugin och att koppla detta mot guin.

Utöver detta har jag supportat andra med deras kodproblem, granskat
alla andras PR och deltagit starkt i gruppdiskussioner om vad som ska
göras.
