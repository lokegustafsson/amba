\subsection*{Projektplanering}
\begin{itemize}
	\item Skrev och reviderade syfte samt hela metodavsnittet.
\end{itemize}

\subsection*{Teknisk design och kod}
\begin{itemize}
	\item Var delaktig i beslutstagande om viktiga design beslut såsom
	      systemarkitektur, beslut om symbolisk exekveringsmotor vs bygga egen
	      symbolisk exekveringsmotor, etc.

	\item Prioritering av grafrendering. Beroende på hur mycket som är kvar att
	      rendera för en given graf prioriteras nästa graf.

	\item Undersökte och skrev kod för IPC som använde protobuf, något vi senare
	      valde att avveckla och ersätta med unix-sockets och serde istället för
	      protobuf för serialisering.

	\item Undersökte om QMP (QEMU Machine Protocol) var rimligt (detta gjorde även Enaya) att använda,
	      men bortprioriterades för 'vanlig' IPC då protokollet var bristfälligt och
	      saknade den funktionalitet vi sökte.
\end{itemize}

\subsection*{Slutrapporten}
\begin{itemize}
	\item Revidering av syfte, inledning, existerande verktyg, etc.

	\item Delar i \textit{2.3 Symbolisk exekvering}: exempelfigurer, figurtext, ytterligare
	      förklaringar om symbolisk exekvering, etc.

	\item Hela \textit{2.4 Fuzzing}.

	\item Hela \textit{6. Evaluering}.
\end{itemize}

Vi har inte haft några tydliga roller, utan skapat uppgifter baserat på vad som
behövdes göras avseende tidsplanen. Den enda rollen som roterat något är rollen
som sekreterare som varierat främst mellan mig, Enaya och Samuel, men Enaya har främst
besuttit denna roll.

Sammanfattningsvis har vi alla samarbetat för att ta oss igenom olika delar av
rapporten där även tekniska beslut, antingen genom 1on1 diskussioner eller
helgrupp tagits tillsammans. Vi har även haft ett review-system, där vi skapat
ett pull-request i github för en task som sedan reviewats av andra
gruppmedlemmar tills det approvas och läggs till i kodbasen eller rapporten.



