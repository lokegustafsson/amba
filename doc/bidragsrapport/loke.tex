\subsection*{Projektplanering och design}

\begin{itemize}

	\item Jag presenterade projektförslaget för detta kandidatarbetet i höstas. Därmed bidrog jag i
	      början av projektet genom att skriva dokumentation om mina designidéer samt samla ihop det
	      underlag jag behövde för att inför gruppen presentera alternativen: 1. Bygga egen symbolisk
	      exekveringsmotor (annan inriktning på projektet i stort), 2. Använda SymQEMU (bättre prestanda,
	      sämre docs), 3. Använda S2E (bättre docs). I LP3LV2 bestämde vi under ett gruppmöte oss för att
	      välja S2E

	\item Till planeringsrapporten skrev jag stora delar av första utkasten till avsnitten Bakgrund,
	      Problem/Uppgift samt Avgränsningar. Dessa avsnitt tar sammalagt totalt upp ca halva
	      planeringsrapporten (men de putsades alla rätt noga av andra i gruppen också!)

\end{itemize}

\subsection*{Programmering och S2E-bygge}

\begin{itemize}

	\item LP3LV2 till LP3LV8 lade jag ned ca 110h på att bygga och köra S2E. Detta var svårare än vi
	      förväntade oss, pga trasiga och ofullständiga bygginstruktioner. Jag skrev även den största
	      delen av grundkoden i AMBA/huvudprocessen, för att generera S2E-config och starta QEMU.

	\item Jag skrev huvudprocesskomponenterna QMP, IPC, extraherade run\_QEMU till egen tråd och
	      skrev QEMU-gäst-komponenten bootstrap.

	\item Jag integrerade egui i huvudprocessen, skrev grafvy-egui-komponenten och
	      grafutplaceringsalgoritmen. Jag förbättrade successivt sedan grafutplaceringskoden kontinuerligt
	      under projektet.

	\item Slutligen implementerade jag visning av disassembly, addresser och allt annan nodmetadata
	      som visas i GUI:n. Här skrev jag inte alla delar själv, utan använde Albins disassembler, Enayas
	      debugdata och AmbaPlugins existerande arkitektur som flera skrivit, men framförallt Samuel.

\end{itemize}

Sammanfattningsvis, programmeringsmässigt, av koden som används i slutprodukten och inte tagits bort
längs vägen, har jag skrivit: Hela paketeringskoden för S2E, nästan all huvudprocesskod med de
viktiga undantagen: debugdataextraktion ur binär, linjegrafskompression, och starkt anslutna
komponenter. Utöver detta har jag skrivit bootstrap-binären som kör inuti QEMU innan den analyserade
binären samt en relativt liten del av AmbaPlugin.

\subsection*{Slutrapporten}

\begin{itemize}

	\item Av planeringsrapporten är i princip allt jag skrev utbytt, utöver kanske delar av
	      teoriavsnittet "Binäranalys" samt figur 2.4 i teoriavsnittet "Exekveringsmotor".

	\item Hela \textit{3 Existerande Verktyg / 3.4 Visualiserad Symbolisk Fuzzing}.

	\item I princip hela \textit{4 AMBA}.

	\item I princip hela \textit{5 Implementation} undantaget arkitekturdiagrammet (som ändå är
	      handritad placeholder atm). Detta var dock en omskrivning, inte nyskrivning, så en del av
	      innehållet är starkt inspirerat av texten som var där innan jag skrev om.

\end{itemize}
