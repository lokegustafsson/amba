\subsection*{Projektplanering och design}

\begin{itemize}

	\item Jag presenterade projektförslaget för detta kandidatarbetet i höstas. Därför var det
	      rimligt att i början av projektet skrev dokumentation om mina designidéer samt samlade ihop
	      underlag för att inför gruppen presentera alternativen: 1. Bygga egen symbolisk
	      exekveringsmotor, 2. Använda SymQEMU, 3. Använda S2E. I LP3LV2 bestämde vi under ett gruppmöte
	      oss för att välja S2E.

	\item Till planeringsrapporten skrev jag stora delar av avsnitten Bakgrund, Problem/Uppgift samt
	      Avgränsningar.

\end{itemize}

\subsection*{Programmering och S2E-bygge}

\begin{itemize}

	\item LP3LV2 till LP3LV8 lade jag ned ca 110h på att bygga och köra S2E. Detta var svårare än vi
	      förväntade oss, pga trasiga och ofullständiga bygginstruktioner. Jag skrev här även den största
	      delen av grundkoden i AMBA/huvudprocessen, för att generera S2E-config och starta QEMU.\@

	\item Jag skrev huvudprocesskomponenterna QMP (pratar med QEMU), IPC (pratar med
	      AmbaPlugin), QEMU (som startar qemu). Jag skrev också QEMU-gäst-komponenten bootstrap.

	\item Jag integrerade egui i huvudprocessen (trådarna GUI, EMBEDDER, CONTROL), skrev
	      grafvy-egui-komponenten och grafutplaceringsalgoritmen.

	\item Slutligen implementerade jag visning av disassembly, addresser och all annan nodmetadata
	      som visas i GUI:n. Här använde jag vissa färdiga komponenter, som Enayas wrapper kring ett
	      debugdatabibliotek och AmbaPlugins existerande arkitektur (som Samuel+fler skrivit).

\end{itemize}

Sammanfattningsvis, programmeringsmässigt, av koden som används i slutprodukten och inte tagits bort
längs vägen, har jag skrivit: 1.\ Hela paketeringskoden för S2E. 2.\ Nästan all huvudprocesskod med de
viktiga undantagen: debugdataextraktion ur binär, linjegrafskompression, och starkt anslutna
komponenter. 3.\ bootstrap-binären som kör inuti QEMU innan den analyserade binären samt en relativt
liten del av AmbaPlugin.

\subsection*{Slutrapporten}

\begin{itemize}

	\item Hela \textit{3.4 Visualiserad Symbolisk Fuzzing}.

	\item I princip hela \textit{4 AMBA}.

	\item I princip hela \textit{5 Implementation}. Detta var dock en omskrivning, inte nyskrivning,
	      så en del av innehållet är starkt inspirerat av texten som var där innan jag skrev om.

\end{itemize}
