\documentclass[12pt]{article}

%%%%%%%%%%%%%%%%%%%%%%%%%%%%%%%%%%%%%% PACKAGES %%%%%%%%%%%%%%%%%%%%%%%%%%%%%%%%
\usepackage[swedish]{babel}
\usepackage{hyperref}
\usepackage{pgfgantt} % Gantt charts
\usepackage{pdflscape} % Landscape mode

% TODO remove these packages for final report
%%%%%%%%%%%%%%%%%%%%%%%%%%%%%%%% "Debug" packages %%%%%%%%%%%%%%%%%%%%%%%%%%%%%%

% Comments in the margin, use \todo{} to add comment to margin.
% Docs: http://tug.ctan.org/macros/latex/contrib/todonotes/todonotes.pdf
\usepackage[colorinlistoftodos]{todonotes}

%%%%%%%%%%%%%%%%%%%%%%%%%%%%%%%%%  TITLE PAGE  %%%%%%%%%%%%%%%%%%%%%%%%%%%%%%%%%

\title{Projektrapport - Dekompilering med ungefärlig symbolisk exekvering}
\author{
    Albin Otterhäll\thanks{GU-IT} \and
    Clara Salber\thanks{GU-IT} \and
    Enayatullah Norozi\thanks{TKDAT} \and
    Linus Wallman\thanks{TKITE} \and
    Loke Gustafsson\thanks{TKTEM} \and
    Samuel Kyletoft\footnotemark[3]
}

%%%%%%%%%%%%%%%%%%%%%%%%%%%%% DOCUMENT STRUCTURE %%%%%%%%%%%%%%%%%%%%%%%%%%%%%%%

\begin{document}
\maketitle

\newpage

% TODO remove this for final report
\listoftodos
\newpage

\tableofcontents
\newpage

Vad repporten ska innehålla:
\url{https://chalmers.instructure.com/courses/22323/assignments/66457?module_item_id=337856}.

\section{Bakgrund}
% BAKGRUNDENS UPPGIFT?: Motivera behovet att visualiseringsverktyg för fuzzing
% (fuzzing används slarvigt, inkludera tex concolic testing)

Vi läser ofta källkod för att förstå program, men ibland är det gynnsamt att istället betrakta
maskinkoden direkt. Detta kan vara för att
\begin{itemize}
  \item utesluta påverkan av kompilatorbuggar som ger oväntad maskinkod
  \item se hur undefined behavior (UB) har utnyttjats av kompilatorn
  \item källkoden inte är tillgänglig
\end{itemize}

Det finns en uppsjö av metoder vi kan använda för att analysera en exekverbar binär. Vi kan bland
annat
\todo{källhänvisa}
\begin{enumerate}
  \item disassemblera binären och läsa dess funktioner för att förstå vad de gör.
  \item dekompilera assemblykoden med ett verktyg som ger högnivåpseudokod, som vi kan läsa.
  \item köra binären på speciella testfall och jämföra svaret med vad vi förväntar oss. Om
    programmet implementerar en specifikation kan vi använda en existerande testsamling.
  \item fuzztesta binären, alltså automatiskt generera testfall tills ett orsakar en crash eller
    annat dåligt beteende i binären. Många fuzztestmotorer skapar testfall med en evolutionär
    algoritm, och många använder instrumentering över vilka programhopp som tas för att bedöma
    testfalls nyttighet.
  \item använda concolic testing, alltså fuzzing där en SMT solver genererar nya testfall genom att
    lösa för testfall som orsakar annorlunda programhopp.
  \item stega igenom programmet i en debugger för att se exakt vad programmet gör med viss input.
\end{enumerate}

För att förstå programmet allmänt behöver vår förståelse vara både \textit{korrekt} och
\textit{abstrakt}, där vi med \textit{korrekt} menar att vi aldrig drar felaktiga slutsatser och med
\textit{abstrakt} menar att vi kan resonera om programmet generellt i motsats till att resonera om en
specifik konkret indata i taget.

Metod 1-2, att läsa kod, kan ge oss en \textit{abstrakt} förståelse av vad programmet gör, men för
att verifiera att vi inte resonerat fel behöver vi kunna testa hypoteser vilket kräver att vi köra
programet. Vi kan inte bilda en \textit{korrekt} förståelse genom att enbart läsa kod.

Metod 3-5, att köra programmet på testfall, ger oss framförallt en black-box-förståelse av
programmet. Att vi har tillgång till binären och exekveringsmiljön används endast som ett verktyg
för att generera nya testfall. Fuzzing och concolic testing kan köras helautomatiskt och är
\textit{korrekta}. Men ofta är en tillräckligt täckande sökning av indatarummet omöjlig, och då kan
den automatiska analysen ha missat ett kvalitativt annorlunda beteende. Dessutom ger en omfattande
uppsättning indata-utdata-par inte användaren samma information som källkoden ger. Därmed är
helautomatiska analysmetoder inte \textit{abstrakta}. Notera att det inte nödvändigtvis tyder på en
brist i den automatiska analysen att ett kvalitativt annorlunda beteende missas, för det gömda
beteendet skulle kunna vara en konsekvens av komplicerad kod, som till exempel ett hoppvilkor
beroende på en kryptografisk hash av indatan. Men en analysmetod borde kunna peka ut var dess
förståelse tar slut, snarare än att utelämna detta fullständigt vilket är vad avsaknaden av testfall
visar sig som.

Med metod 6, i en debugger, kan användaren följa exekveringen för en viss indata utan att riskera
att missförstå hur datan transformeras. Om användaren har ett oändligt tålamod kan de göra detta om
och om igen för olika indata genererade med till exempel fuzzing. Varje genomstegning ger
information om koden som behandlar indatan men också viss information om övrig kod -- till exempel
kan ett svårtaget hopp indikera en plats för användaren att rikta sin uppmärksamhet mot. Detta ger
en både \textit{korrekt} och \textit{abstrakt} förståelse, men med en orimlig manuell arbetsbörda
för användaren.

En helautomatisk \textit{korrekt} metod kan ge en \textit{abstrakt} förståelse om processens förlopp
visualiseras för användaren. Valet mellan manuell arbetsbörda som ger djup förståelse och en
testfallsgenerationsdriven process som ger översiktlig förståelse kan genomföras av användaren om
verktygen stödjer hela spektrat.


\section{Syfte}
Projektets mål är att utveckla ett binäranalysverktyg som, utan tillgång till
källkoden, möjliggör effektiv analys av binära program genom att kombinera
fördelarna av både automatiska och manuella analyser. Fokuset ligger på
kraftfulla analyser som hoppas kunna kombineras med ett intuitivt
användargränssnitt för att vara tilltalande till en mångsidig målgrupp,
inklusive säkerhetsforskare och programvaruutvecklare. Ett grafiskt
användargränssnitt är tänkt att sätta den mänskliga användaren i fokus och ge
en bättre programförståelse än automatiska analysverktyg. Verktyget ska bidra
till en högre förståelse av binära program och säkerhet i tekniska system.

\todo[inline]{Vad är nyttan i just vårt projekt? Hur skiljer vårt projekt sig från andra verktyg?}


\section{Problem/Uppgift}
% Det här avsnittet är ofta den viktigaste delen av planeringsrapporten
% (och av den slutgiltiga uppsatsen/rapporten). Den syftar till att
% identifiera frågan/frågorna som ska tas upp i projektet. Det är
% viktigt att gruppen gör en problemanalys även om det i
% projektförslaget redan finns ett problem (en uppgift)
% specificerat. Anledningen till detta är att det riktiga primära
% problemet ofta skiljer sig från det i början av
% uppdragsgivaren/förslagsställaren/kunden föreslagna. Problemanalysen
% syftar också till att bryta ner problemet/uppgiften i mindre och mer
% detaljerade delproblem/deluppgifter, vilket också leder till
% formulering av delsyften. Genom att göra detta får studenterna mycket
% bättre förståelse för de olika aspekterna av problemet/uppgiften. Utan
% den här informationen är det omöjligt att identifiera vilken
% information som behövs, vilka informationskällor som behövs och
% lämpliga tillvägagångssätt.

% En bra problemanalys som identifierar delproblem/deluppgifter och
% delsyften vilar i många fall på användning av teorier och modeller
% från litteraturen. En litteraturgenomgång bör därför genomföras tidigt
% i processen.

Symbolic execution engine
S2E
state-merging


\section{Avgränsningar}

\subsection{Introduktion} 

Projektet avgränsas i och med att existerande verktyg (S2E) kommer användas 
istället för att bygga en symbolic execution engine från grunden. 
Att använda S2E innebär att arbetet avgränsas till att skapa plugins som 
bygger ut motorn. Varken emulator eller motor ska byggas och de uppgifter 
som ingår i att skapa en dekompilator exkluderas. 

\subsection{Begränsningar} 

Belsutet innebär att applikationens utformning blir bunden till 
verktygens tekniska begränsningar. Exempelvis försvinner möjligheten att 
förbättra state-merging. 

\subsection{Inriktning} 

Avgränsningen medför att fokus flyttas ifrån motorns tekniska detaljer 
till att utveckla en användbar slutprodukt som bygger ut S2E's redan 
existerande funktionalitet med ett grafiskt användargränsnitt 
och möjlighet att analysera och interaktivt besluta om det dekompilerade 
programmets kontrollflöde under exkevering.
 


\section{Metod/Genomförande}
\subsection{Val av strategi} 
För att uppnå och redogöra för möjligheterna i en
potentiell applikation reflekterades det över två möjliga alternativ att
fullfölja. Dels diskuterades alternativet att bygga en symbolic-execution engine
från grunden och fokusera på dess tekniska detaljer och dels att använda en
existerande produkt där fokuset istället ligger på att bygga plugins vars syfte
är att utöka den existerande motorn med fler funktioner. 

Med syfte att göra framsteg valdes det att kolla vidare hur S2E kan hjälpa i
frågan om att utöka funktionalitet i en existerande motor. S2E är utvecklat
till större del i programspråket C/C++ och med anledning av valet att utveckla i ett
annat programmeringsspråk, det vill säga Rust, krävs implementation av
C/C++-bindings. I ett första steg valdes det därför att se över hur detta går till
på lämpligt sätt och hur detta kan automatiseras med hjälp av andra verktyg
där bland annat autocxx ansågs som ett lovande alternativ. 

\subsection{Tillvägagångssätt} 
I avsikt att uppnå syftet med projektet, mer
specifikt att utveckla en applikation, bestämdes det att utveckla demos av enkla
pseudo-c program med hjälp av bindings till S2E som implementerats för att
undersöka potentiella användningsområden och utveckla dessa vidare. 

% utveckla detta (konkretisera etc.)
För att vidare kunna bestämma huruvida en demo faller inom ett lämpligt
användningsområde kommer en avgränsning ske efter vad som är rimligt; hur väl
kan det komma att tillämpas i en applikation; vad är relevant att undersöka med
applikationen samt om det är genomförbart inom satt tidsram. 

I ett senare och/eller parallellt steg ska ett intuitivt grafiskt användargränssnitt
utvecklas som tillåter användaren att traversera genom applikationen och göra
egna beslut gällande branches etc. där användaren själv bestämmer interaktivt
nästa beslut som görs. 

% Hur är detta kopplat till vårt syfte? Hur uppnår vi syftet med rapporten genom
% vald metod?

% (varför är typ besvarat i val av strategi-avsnittet)

% Hur besvarar vi huruvida valet av strategi är rimligt?

%  



\section{Samhälleliga och etiska aspekter, bedömning om det behöver beaktas för vald problemställning}
% Lärandemål i kursplanen för kandidatarbetet:
% "bedöma om samhälliga och etiska aspekter behöver beaktas för vald problemställning och där det är relevant, analysera dessa aspekter i uppsatsen/rapporten

% Beslutsmodell för kritiskt tänkande om etiska frågor
% Denna modell är tänkt att användas så att gruppen går igenom frågorna en gång och svara på dem preliminärt.
% När gruppen gjort detta kan de gå igenom frågorna igen för att fördjupa analysen.

Vi har här använt 

Att \emph{komprementera} ett datorsystem innebär att man på något sätt skadar ett datorsystems konfidentialitet; tillgänglighet; eller integritet.
Med \emph{försvarare} syftar vi på de personer som har till uppgift att förhindra att ett datorsystem blir komprementerat.
Med \emph{attackerare} syftar vi på de personerna som har som mål att komprementera datorsystem.

\subsection{Q1: Vilka etiska aspekter (värden) är relevanta för projektet?}

% Det finns några få centrala etiska aspekter vilka alltid är viktiga att undersöka om de är relevanta – och om så är fallet – uppfylla dessa.
% Dessa är att vi inte ska göra skada, vi ska göra nytta, och vi ska inte inskränka på andras autonomi och integritet.
% Att göra nytta ska här tolkas brett: både den inomvetenskapliga och utomvetenskapliga nyttan är relevant.
% Till exempel kan det vara så att ett projekt inte har någon specifik nytta för samhället men att det ändå finns goda skäl att genomföra det eftersom det skulle tillföra något intressant och relevant till grundforskningen.
% Beroende på projektet kan det finnas andra relevanta aspekter att ta hänsyn till.

En stor del av teknikutveckling av verktyg som har som mål att hjälpa försvarare med att hitta säkerhetsbrister i deras programvarar kan även användas av attackerare.
Medan försvararna använder verktygen för att hitta brister för att de ska veta vilka åtgärder de ska vidta för att höja säkerheten, använder attackerarna verktygen för att hitta brister som de sedan kan uttnytja för att påverka datorsystem negativt.
Alltså använder försvarare och attackerna verktygen på liknande sätt, men vad de sedan gör med den informationen skiljer dem åt.

- Ökad förståelse för hur program exekverar. Det gör det möjligt att hitta buggar i program, eller optimera dem.

Ett möjligt problem är att verktygen som utvecklas för försvararna blir såpass bra att attackerarna helt börjar använda dem för att utföra attacker.

\subsection{Q2: Hur kan vi genomföra vårt projekt för att undvika etiska problem med vår metod?}

% Givet att gruppen ska genomföra ett visst projekt kan det finnas en rad olika sätt som det kan utföras på där vissa genomföranden är mer problematiska än andra.
% Ett exempel är ett projekt där studenternas frågeställningar kan undersökas med djurförsök.
% Här bör diskuteras om djurförsöken kan ersättas med andra typer av försök, alternativt använda färre djur, göra försöken mindre plågsamma och så vidare.
% Ett annat exempel är ett projekt har som mål att utveckla en teknisk lösning för att minska människors ångestproblematik.
% Studenterna har tänkt testa denna lösning på sina vänner och bekanta.
% I ett sådant projekt är det viktigt att vara medveten om och diskutera att metoden kan medföra problem för deltagarnas välbefinnande.

Vi ser inte några skador som man upkomma under utvecklandet av projektet.

\subsection{Q3: Vad kan det finnas för nytta eller etiska problem med det sannolika resultatet (utfallet) av projektet som man bör ta hänsyn till?}

% När projektet är genomfört kan det bidra med nytta till både forskning och samhälle.
% Det är viktigt att beskriva nyttan i konkreta termer och också beskriva om projektets färdigställande riskerar att leda till skador på olika sätt.
% Ett exempel är ett projekt som genomförs i en stadsdel med målet att öka tryggheten och delaktigheten för de boende genom en boendedriven innovation, där man bör fundera över vad som troligt händer efter det att projektet avslutas.

Ett potentiellt etiskt problem med projektets eventuella resultat är om vi testar att använda våra verktyg på ett program som har användare.
Det kan medföra att vi då besitter kunskaper som kan utnyttjas för skada användare av programvaran.
Vi kommer då att hantera det problemet genom att genomföra ett ansvarsfullt avslöjande (engelska: responsible disclousre).
Det exakta processen som vi kommer att använda kommer att beslutas om situationen uppkommer, men Googles process är välkänd. https://about.google/appsecurity/

\subsection{Q4. Vilka berörs av projektets genomförande eller av det sannolika resultatet (utfallet) av projektet? Hur berörs de? Finns det etiska problem kopplat till detta som man bör ta hänsyn till?}
% Vid en etisk analys av ett projekt är det av yttersta vikt att fråga sig vilka som berörs av projektet samt på vilket sätt de påverkas.
% Till exempel, om ett projekt syftar till att genmodifiera grödor så att de blir mer resistenta mot bekämpningsmedel så kan en effekt av detta bli när dessa grödor kommer ut på marknaden att de bönder som arbetar med dessa grödor i fattigare delar av världen tar stor (ekonomisk och/eller fysisk) skada av detta.
% Eftersom skador på redan utsatta grupper kan bli extra allvarliga från ett etiskt perspektiv, bör denna sorts överväganden få stor vikt.

Attackerare som får ett nytt verktyg att använda för att attackera
Försvarare som får ett nytt verktyg att använda för att försvara system
Användarei kan drabbas om attackerarna lyckas använda verktyget för att till en mindre kostnad hitta en sårbarhet för att sedan kompromitera systemet.

\subsection{Q5. Vad bör vi göra om vi inte hittat några relevanta etiska aspekter (värden) rörande projektet?}

% Om studentgruppen har gått igenom steg 1-4 ovan och analyserat sitt projekt och projektets möjliga effekter utan att hitta några relevanta etiska eller samhälleliga aspekter byter gruppen analysnivå (systemnivå).
% Beroende på vilken nivå projektet analyseras på kan den utom- och inomvetenskapliga relevansen bedömas på olika sätt.
% Till exempel, gruppens projekt är att i slutändan bidra med att tillsätta en extra skalärboson till standardmodellen.
% Gruppens projekt i sig aktiverar antagligen inte några relevanta etiska aspekter i sitt genomförande eller i sitt utfall.
% Ändå går det att tänka sig att resultaten i det större forskningssammanhanget, som studenternas projekt bidrar till, kan ha en rad olika positiva och negativa implikationer för både forskning och samhälle.
% Ett annat exempel kan vara ett projekt med syfte att bidra till effektivare bränsleanvändning av lastbilar som kan leda till mindre utsläpp och billigare drift för det enskilda fordonet, där en högre systemnivå kan vara dieselfordons roll i ett transportsystem där negativa konsekvenser kan vara ökade alternativkostnader för utvecklandet av motorer som inte drivs av fossila bränslen.
% När studenterna bytt analysnivå, går de igenom steg 1 till 4 igen.



\section{Tidsplan}

% \begin{landscape}
\begin{figure}[htp]
\begin{center}

\todo[inline]{Add more things to Tidsplan from Canvas}
\todo[inline]{Make GANTT chart more visually pleasing}

\begin{ganttchart}[
y unit chart=0.6cm,
vgrid={*{8}{gray, dotted}, *1{black, dashed}},
% expand chart=\textwidth,
    ]{1}{18}
    \gantttitle{LP 3}{9} \gantttitle{LP 4}{9} \\
    \gantttitlelist{1,...,9}{1} \gantttitlelist{1,...,9}{1} \\

    \ganttbar{Projektplan}{1}{4} \\

    % Eget
    \ganttbar{*S2E bygge}{2}{3} \\
    \ganttbar{*S2E bindings/infrastruktur}{3}{5} \\
    \ganttbar{*Fler analyskomponenter?}{5}{10} \\
    \ganttbar{*Brainstorming demos}{5}{10} \\
    \ganttbar{*Implementing demos}{5}{12} \\
    \ganttbar{*GUI-ramverk}{6}{9} \\ \\
    \ganttbar{*Slutprodukt}{8}{13} \\

    \ganttbar{Halvtidsredovisning}{6}{8} \\

    % TODO include deadlines for drafts of report for feedback meetings with fackspråk.
    \ganttbar{Film}{16}{17} \\
    \ganttbar{Skriftlig individuell opposition}{18}{18} \\
    \ganttbar{Slutrapport, förgranskning}{14}{14} \\
    \ganttbar{Slutrapport}{16}{16} \\
    \ganttbar{Slutgiltig rapport}{19}{19} \\

    \ganttbar{Slutredovisning}{18}{18} \\

    \ganttlink{elem1}{elem2}
    \ganttlink{elem2}{elem5}
    \ganttlink{elem4}{elem5}
    \ganttlink{elem5}{elem7}
    \ganttlink{elem6}{elem7}

\end{ganttchart}

% \begin{ganttchart}[y unit title=0.4cm,
% y unit chart=0.5cm,
% vgrid,hgrid, 
% title label anchor/.style={below=-1.6ex},
% title left shift=.05,
% title right shift=-.05,
% title height=1,
% bar/.style={fill=gray!50},
% incomplete/.style={fill=white},
% progress label text={},
% bar height=0.7,
% group right shift=0,
% group top shift=.6,
% group height=.3,
% group peaks={}{}{.2}]{24}
%
% % labels
% \gantttitle{Week}{24} \\
% \gantttitle{Monday}{4} 
% \gantttitle{Tuesday}{4} 
% \gantttitle{Wednesday}{4} 
% \gantttitle{Thursday}{4} 
% \gantttitle{Friday}{4} 
% \gantttitle{Saturday}{4} \\
% %tasks
% \ganttbar{first task}{1}{2} \\
% \ganttbar{task 2}{3}{8} \\
% \ganttbar{task 3}{9}{10} \\
% \ganttbar{task 4}{11}{15} \\
% \ganttbar[progress=33]{task 5}{20}{22} \\
% \ganttbar{task 6}{18}{19} \\
% \ganttbar{task 7}{16}{18} \\
% \ganttbar[progress=0]{task 8}{21}{24}
%
% %relations 
% \ganttlink{elem0}{elem1} 
% \ganttlink{elem0}{elem3} 
% \ganttlink{elem1}{elem2} 
% \ganttlink{elem3}{elem4} 
% \ganttlink{elem1}{elem5} 
% \ganttlink{elem3}{elem5} 
% \ganttlink{elem2}{elem6} 
% \ganttlink{elem3}{elem6} 
% \ganttlink{elem5}{elem7} 
% \end{ganttchart}

% \begin{ganttchart}[
%     % expand chart=\textwidth,
%     y unit chart=1cm,
%     vgrid={*{11}{gray, dotted}, *1{black, dashed}},
%     ]{1}{12}
% \gantttitle{2011}{6} \gantttitle{2012}{6} \\
% \gantttitlelist{1,...,6}{2} \\
% \ganttgroup{Group 1}{1}{7} \\
% \ganttbar{Task 1}{1}{2} \\
% \ganttlinkedbar{Task 2}{3}{7} \ganttnewline
% \ganttmilestone{Milestone}{7} \ganttnewline
% \ganttbar{Final Task}{8}{12}
% \ganttlink{elem2}{elem3}
% \ganttlink{elem3}{elem4}
% \end{ganttchart}

% \begin{ganttchart}[
%    vgrid={*{11}{gray, dotted}, *1{black, dashed}},
%    bar label node/.append style={
%      align=left,
%      % text width=width("Aim 2. Software verificationx")
%      text width=110
%     }
%    ]{1}{24}
% \gantttitle{Year 1}{12} \gantttitle{Year 2}{12} \\
% \ganttbar{Migration}{1}{8} \\
% \ganttbar{Software verification}{6}{12} \\
% \ganttbar{Hardware portability}{12}{18} \\
% \ganttbar{Documentation}{8}{24}
% \end{ganttchart}

\end{center}
\caption{Tidsplan}
\end{figure}
% \end{landscape}


\bibliographystyle{plain} % We choose the "plain" reference style
\bibliography{refs} % Entries are in the refs.bib file
\end{document}
