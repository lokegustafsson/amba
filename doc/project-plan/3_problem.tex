% Det här avsnittet är ofta den viktigaste delen av planeringsrapporten
% (och av den slutgiltiga uppsatsen/rapporten). Den syftar till att
% identifiera frågan/frågorna som ska tas upp i projektet. Det är
% viktigt att gruppen gör en problemanalys även om det i
% projektförslaget redan finns ett problem (en uppgift)
% specificerat. Anledningen till detta är att det riktiga primära
% problemet ofta skiljer sig från det i början av
% uppdragsgivaren/förslagsställaren/kunden föreslagna. Problemanalysen
% syftar också till att bryta ner problemet/uppgiften i mindre och mer
% detaljerade delproblem/deluppgifter, vilket också leder till
% formulering av delsyften. Genom att göra detta får studenterna mycket
% bättre förståelse för de olika aspekterna av problemet/uppgiften. Utan
% den här informationen är det omöjligt att identifiera vilken
% information som behövs, vilka informationskällor som behövs och
% lämpliga tillvägagångssätt.

% En bra problemanalys som identifierar delproblem/deluppgifter och
% delsyften vilar i många fall på användning av teorier och modeller
% från litteraturen. En litteraturgenomgång bör därför genomföras tidigt
% i processen.

Symbolic execution engine
S2E
state-merging
