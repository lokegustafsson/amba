% Create a beamer file with cool style.

\documentclass{beamer}
\usetheme{Madrid}
\usecolortheme{seahorse}
\usefonttheme{serif}
\useinnertheme{rectangles}
\useoutertheme{infolines}

\usepackage[swedish]{babel}
\usepackage[utf8]{inputenc}
\usepackage{
    xstring,
    xspace,
}

\begin{document}

\title{AMBA}
\subtitle{Interaktiv visualisering av symbolisk fuzzing}
\author{
	\mbox{Loke Gustafsson} \and
	\mbox{Samuel Kyletoft} \and
	\mbox{Enayatullah Norozi} \and
	\mbox{Albin Otterhäll} \and
	\mbox{Clara Salberg} \and
	\mbox{Linus Wallman}
}
\date{\today}

\frame{\titlepage}


% 1. (minut 1) Motivera behovet att analysera en binär (saknar källkod,
% kompilatorbuggar, ditt språk har sämst tooling)
\begin{frame}
	% Program analysis is vital for verifying program properties. This
	% requires, as you might imagine, examining many paths a program may take,
	% all of them if you want the best results.

    \textbf{Program analysis}
	\vspace{3.0mm}
	\begin{figure}
		\includegraphics[height=5cm]{assets/DALL·E 2023-05-21 23.06.07 - magnifying glass zooming in on a hello world program written in C.png}
		\footnote{\tiny Generated by DALL$\cdot$E}
	\end{figure}
\end{frame}

\begin{frame}
	\vspace{3.0mm}
    \textbf{Program analysis}
    \begin{center}
			\tiny{
                \begin{tikzpicture}[
                    roundnode/.style={circle, minimum size=7mm},
                    question/.style={diamond, draw, fill=lightgray!80},
                    squarednode/.style={draw, fill=lightgray!80, minimum height=0.3cm},
                    arrow/.style={draw, -latex'}
                    ]

					\node [squarednode] (A)                       {inp $\leftarrow$ input()};
					\node [question] (B)       [below=of A]  {inp = 93784739};
                    \node [squarednode] (C)     [right =of B] {...};
                    \node [squarednode] (E)     [below=of C] {x $\leftarrow$ 1 / (inp-93784739)};
                    \node [roundnode, draw=red!60, fill=red!5] (F)     [below=of E] {exitcode 1};
                    \node [squarednode]    (D)    [below=of B]  {...};
                    \node [roundnode, draw=green!60, fill=green!5]    (G)    [below=of D]  {exitcode 0};

					\draw [arrow] (A) -- (B);
                    \draw [arrow] (B.east) -- (C) node [midway, above] (yesCond) {yes};
					\draw [arrow] (B.south) -- (D) node [midway, left] (noCond) {no};
					\draw [arrow] (C) -- (E);
					\draw [arrow] (D) -- (G);
					\draw [arrow] (E) -- (F);

				\end{tikzpicture}
			}
    \end{center}
\end{frame}

\begin{frame}
	\vspace{3.0mm}
    \textbf{Program analysis}
    \begin{center}
			\tiny{
                \begin{tikzpicture}[
                    roundnode/.style={circle, minimum size=7mm},
                    question/.style={diamond, draw, fill=lightgray!80},
                    squarednode/.style={draw, fill=lightgray!80, minimum height=0.3cm},
                    arrow/.style={draw, -latex'}
                    ]

					\node [squarednode] (A)                  {inp $\leftarrow$ input()};
					\node [question] (B)       [below=of A]  {inp = 93784739};
                    \node [squarednode] (C)     [right =of B] {...};
                    \node [squarednode] (E)     [below=of C] {x $\leftarrow$ 1 / (inp-93784739)};
                    \node [roundnode, draw=red!60, fill=red!5] (F)     [below=of E] {exitcode 1};
                    \node [squarednode]    (D)    [below=of B]  {...};
                    \node [roundnode, draw=green!60, fill=green!5, very thick]    (G)    [below=of D]  {exitcode 0};

					\draw [arrow] (A) -- (B);
                    \draw [arrow] (B.east) -- (C) node [midway, above] (yesCond) {yes};
                    \draw [arrow] (B.south) -- (D) node [midway, left, blue] (noCond) {\textbf{no}};
					\draw [arrow] (C) -- (E);
					\draw [arrow] (D) -- (G);
					\draw [arrow] (E) -- (F);

				\end{tikzpicture}
			}
    \end{center}
\end{frame}


\begin{frame}
	\vspace{3.0mm}
    \textbf{Program analysis}
    \begin{center}
			\tiny{
                \begin{tikzpicture}[
                    roundnode/.style={circle, minimum size=7mm},
                    question/.style={diamond, draw, fill=lightgray!80},
                    squarednode/.style={draw, fill=lightgray!80, minimum height=0.3cm},
                    arrow/.style={draw, -latex'}
                    ]

					\node [squarednode] (A)                       {inp $\leftarrow$ input()};
					\node [question] (B)       [below=of A]  {inp = 93784739};
                    \node [squarednode] (C)     [right =of B] {...};
                    \node [squarednode] (E)     [below=of C] {x $\leftarrow$ 1 / (inp-93784739)};
                    \node [roundnode, draw=red!60, fill=red!5, very thick] (F)     [below=of E] {exitcode 1};
                    \node [squarednode]    (D)    [below=of B]  {...};
                    \node [roundnode, draw=green!60, fill=green!5]    (G)    [below=of D]  {exitcode 0};

					\draw [arrow] (A) -- (B);
                    \draw [arrow] (B.east) -- (C) node [midway, above, blue] (yesCond) {\textbf{yes}};
					\draw [arrow] (B.south) -- (D) node [midway, left] (noCond) {no};
					\draw [arrow] (C) -- (E);
					\draw [arrow] (D) -- (G);
					\draw [arrow] (E) -- (F);

				\end{tikzpicture}
			}
    \end{center}
\end{frame}

\begin{frame}
	% Binary analysis is program analysis on the binary and is in many cases more
	% useful then program analysis on the source code. In some cases you don't
	% have a choice, like malware analysis where there is no source code, and in
	% other cases, because it shows the true behaviour of the program, after
	% potential compiler bugs introduced and other semantic effects applied.
	% Finally, binary analysis is a more general method for many software
	% stacks, in contrast to language specific, source level analysis tools.

	\textbf{Why binary analysis?}
    \vspace{1.8mm}
	\begin{itemize}
		\item Missing source code
		\item Compiler introduced bugs
		\item Semantic gap between source code and binary
		\item Doesn't depend on source language
	\end{itemize}
\end{frame}



% 2. (minut 2-3) Förklara några binäranalysmetoder (debugger, fuzzing)

\begin{frame}
	% Some examples of binary analysis methods are: 
	% binary debugging, which might be a little too manual in many cases;

	% Studying disassembly or decompiling using heuristics and meta data to
	% generate a higher level pseudocode;

	% fuzzing, generating different inputs and studying the outcome of the
	% program, like a crash;

	% Symbolic fuzzing, which we will get to.

	% All methods have their pros and cons and are applicable in different situations.
	% Binary debugging and studying disassembly or decompilation could be very
	% manual and burdenful for the analyst, but it could also give a very good
	% abstract understanding of the program.

	% Fuzzing is an automatic method and requires no interaction, but it doesn't
	% give any abstract understanding of the program. Only determining some
	% inputs satisfying user defined objectives. Like determining crashes,
	% detecting memory leaks and so on.

	% Another problem with fuzzing is the testcase generation. It is not trivial
	% how to generate test cases which effectively covers a large set of
	% execution paths so that no same paths are unneccessarily tested multiple
	% times, or reachable paths that were not explored.

	% The specific problem of test case generation is solved by symbolic fuzzing
	% which relies on symbolic execution.

	\begin{columns}[t]
		\begin{column}{0.5\textwidth}
			\begin{itemize}
				\item Binary debugging
				\item Disassembly
				\item Decompilation
				\item Fuzzing
				\item Symbolic fuzzing
			\end{itemize}
		\end{column}
		\begin{column}{0.5\textwidth}
			\includegraphics[width=0.3\textwidth]{assets/GDB_Archer_Fish_by_Andreas_Arnez.svg.png}
			\footnote{\tiny by Andreas Arnez, \href{https://creativecommons.org/licenses/by-sa/3.0/us/deed.en}{CC BY-SA 3.0 us}}

			\includegraphics[width=\textwidth]{assets/disassembly.png}

			\includegraphics[width=\textwidth]{assets/decompil.png}

			\tiny{
				\begin{tikzpicture}

					\node [draw, fill=lightgray!80, minimum height=0.3cm]
					(first) {fuzz data generation};

					\node [draw, fill=lightgray!80, minimum height=0.3cm, right=0.2cm of first]
					(second) {execution};

					\node [draw, fill=lightgray!80, minimum height=0.3cm, below=0.2cm of second]
					(third) {analysis};

					\node [diamond, draw, fill=lightgray!80, minimum height=0.1cm, below=0.3cm of third]
					(fourth) {finished?};

					\node [draw, fill=lightgray!80, minimum height=0.3cm, right=0.2cm of fourth]
					(fifth) {report};

					\path [draw, -latex'] (first) to (second);
					\path [draw, -latex'] (second) to (third);
					\path [draw, -latex'] (third) to (fourth);
					\path [draw, -latex'] (fourth) to (fifth);
					\path [draw, -latex'] (fourth) -| (first);

				\end{tikzpicture}
			}
		\end{column}
	\end{columns}
\end{frame}


% 3. (minut 4) Förklara symbolisk exekvering
\begin{frame}
	\begin{columns}[t]
		\begin{column}{0.4\textwidth}
			\textbf{Symbolic Execution}
			\small
			\begin{enumerate}
				\item Explore possible execution paths
				\item Input as symbolic values
                \item Tracking constraints into conditional expressions
				\item Path explosion
				\item Concolic testing / Symbolic fuzzing
			\end{enumerate}
		\end{column}
	\end{columns}
\end{frame}


% 4. (minut 5-6) Säga att vi gjort detta på maskinkodsnivå med S2E
\section{AMBA}

\subsection{Implementation}
\subsection{Komprimering}

\subsection{Starkt anslutna komponenter}



% 5. (minut 7-11) Demo med enklare, successivt mer komplicerade exempel
\begin{frame}
	\centering
	\huge
	DEMO
\end{frame}


% 6. (minut 12) Utvecklingspotential
\begin{frame}

	\centering
	AMBA is cool, but not useful

\end{frame}

\begin{frame}

	\centering
	AMBA is cool, but not (yet) useful

\end{frame}



% 7. (minut 13) Arbetsprocessen (bortförklara varför utveckingspotential finns)
% (ish vecka för vecka, med figur)
\begin{frame}
	\begin{columns}[t]
		\begin{column}{0.4\textwidth}
			\textbf{Plan}
			\begin{enumerate}
				\item Built \stoe{} week 2--4
				\item Prototyped functionality week 5--13
				\item Integrated in AMBA week 8--16
			\end{enumerate}
		\end{column}
		\begin{column}{0.4\textwidth}
			\pause{}
			\textbf{Reality}
			\begin{enumerate}
				\item Built \stoe{} week 2--7
				\item Prototyped functionality week 6--8
				\item Functionality built directly in AMBA week 8--18
			\end{enumerate}
		\end{column}
	\end{columns}
	\pause{}
	\textbf{Consequences}
	\begin{enumerate}

		\item Features abandoned after investing nontrivial effort (e.g.
		      \texttt{malloc}/\texttt{free} tracking)

		\item Features chosen for simplicity of implementation, not user utility
		      (e.g.\ double-click to prioritize a state subtree)

		\item No external user testing or feedback whatsoever

	\end{enumerate}
\end{frame}


\end{document}
