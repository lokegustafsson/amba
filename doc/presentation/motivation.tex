\begin{frame}
	% Program analysis is vital for verifying program properties. This
	% requires, as you might imagine, examining many paths a program may take,
	% all of them if you want the best results.

	\frametitle{Program analysis}
	\begin{figure}
		\includegraphics[height=5cm]{assets/DALL·E 2023-05-21 23.06.07 - magnifying glass zooming in on a hello world program written in C.png}
		\footnote{\tiny Generated by DALL$\cdot$E}
	\end{figure}
\end{frame}

\begin{frame}
	% Binary analysis is program analysis on the binary and is in many cases more
	% useful then program analysis on the source code. In some cases you don't
	% have a choice, like malware analysis where there is no source code, and in
	% other cases, because it shows the true behaviour of the program, after
	% potential compiler bugs introduced and other semantic effects applied.
	% Finally, binary analysis is a more general method for many software
	% stacks, in contrast to language specific, source level analysis tools.

	\frametitle{Why binary analysis?}
	\begin{itemize}
		\item Missing source code
		\item Compiler introduced bugs
		\item Semantic gap between source code and binary
		\item Widely applicable
	\end{itemize}
\end{frame}
