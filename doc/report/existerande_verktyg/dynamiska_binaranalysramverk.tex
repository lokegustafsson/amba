En metod för dynamisk binäranalys är symbolisk exekvering. Eftersom exekveringen är symbolisk är 
det möjligt att utforska alla möjliga exekveringsvägar i programmet, även de som inte är möjliga
med konkreta inputs. Både \stoe{} och Symqemu är kraftfulla verktyg för att analysera binära program 
med symbolisk exekvering.

SymQEMU är byggt som en förlängning av QEMU, och använder en kombination av dynamisk binär 
översättning och symbolisk exekvering för att analysera binära program som körs inuti emulatorn~\cite{symqemu}.
SymQEMU är främst inriktat på att analysera x86- och ARM-binärer och har begränsade instrumenteringsmöjligheter~\cite{symbexec}.

\stoe{} är ett modulärt bibliotek som ger virtuella maskiner symbolisk exekvering. \stoe{} erbjuder 
redskap för att fokusera utforskningen på delkomponenter av systemet. \stoe{} gör det även 
möjligt för användare att sätta in kod i målsystemet vid specifika punkter under 
exekvering. Det gör att användare kan anpassa analysprocessen efter sina behov~\cite{s2e}. 

Ett annat verktyg för att identifiera buggar och sårbarheter med hjälp av dynamisk symbolisk 
exekvering är SAGE (Scalable Automated Guided Execution).
SAGE var det första verktyget som utförde dynamisk symbolisk exekvering på x86-binärnivå och använder flera 
avgörande optimeringar för att hantera stora exekverings-spår (jmf. engelska execution traces).
För att skala upp till stora exekverings-spår använder SAGE flera tekniker för att förbättra hastigheten och 
minnesanvändningen för constraint generering, exempelvis så kartläggs ekvivalenta symboliska uttryck till samma 
objekt och constraints som redan är tillagda hoppas över~\cite{sage}.