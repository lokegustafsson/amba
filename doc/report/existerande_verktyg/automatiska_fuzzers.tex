Ytterligare en metod för binäranalys är fuzzing. Fuzzing innebär att testa ett program med en
stor mängd olika input. Målet är att simulera oväntade beteenden eller krashar om potentialen för
dessa existerar. Fuzzing är därför ett effektivt medel för att undersöka om ett program innehåller
sårbarheter~\cite{rawat}.

AFL++ är en utveckling av den ursprungliga AFL (American Fuzzy Lop) fuzzern och har
förbättrats med fler funktioner och bättre prestanda. AFL använder täckningsstyrd fuzzing
(jmf. engelska coverage-guided fuzzing) för att generera testfall som är troliga att utlösa buggar
och sårbarheter. AFL spårar kodtäckningen vid varje testfall och använder denna informationen för att
styra hur tidigare testfall mutateras till nya. Även AFL++ muterar input för att generera nya inputs
från befintliga, och använder feedback från programvaran som fuzzas för att styra
fuzzing-processen~\cite{aflplusplus}.
