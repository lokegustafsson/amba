Ghidra är ett \emph{reverse engingeering} ramverk utvecklat av USA:s NSA
(National Security Agency) och kan disassembla en binär till pseudo-C-kod.
Ghidra har också en debugger och funktionsgraf. Debuggern ska underlätta binär
debugging genom att integrera med andra funktioner i Ghidra. Funktionsgrafen
låter användaren se hur programmet är uppbyggt visuellt och hur olika
funktioner interagerar med varandra. Funktionaliteter kan utökas eller andra
funktioner utvecklas genom plugins till Ghidra\cite{ghidra_use_cases}. Ghidra
tillåter även automatisering genom att skriva skript. Ett exempel är ett skript
som hittar exempelvis sårbarhet i form av funktionsanrop till potentiellt
osäkra API-anrop genom statisk analys\cite{ghidra_script}.
