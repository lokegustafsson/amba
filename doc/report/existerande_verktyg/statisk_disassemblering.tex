En vanlig statisk binäranalys är statisk disassemblering (ofta bara kallat
disassembering)~\cite{andriesse2018}. Disassembering innebär att man försöker
återskapa assemblerinstruktioner frän maskinkoden i binären, utan att exekvera
maskinkoden~\cite{andriesse2018}. En av svårigheterna med att översätta
maskinkod till assembler är att särskilja instruktioner från
data~\cite{andriesse2018}.  Till exempel kan man inte veta om maskinkoden
\verb|0x8E| är decimalvärdet $142$ eller x86-instruktionen
\verb|mov|~\cite{andriesse2018}. Det finns i huvudsak två metoder för att
översätta maskininstruktioner till assembler, \emph{linjär disassemblering}; och
\emph{rekursiv disassemblering}~\cite{andriesse2018}.

Ett vanligt disassemblerverktyg är Ghidra, ett program utvecklat av NSA
(National Security Agency)~\cite{ghidra_website}. Ghidra har också en debugger
och funktionsgraf. Debuggern ska underlätta binär debugging genom att
integrera med andra funktioner i Ghidra. Funktionsgrafen låter användaren se
hur programmet är uppbyggt visuellt och hur olika funktioner interagerar med
varandra. Funktionaliteter kan utökas eller andra funktioner utvecklas genom
plugins till Ghidra~\cite{ghidra_use_cases}. Ghidra tillåter även automatisering
genom att skriva skript. Ett exempel är ett skript som hittar exempelvis
sårbarhet i form av funktionsanrop till potentiellt osäkra API-anrop genom
statisk analys~\cite{ghidra_script}. Andra vanliga disassemblerverktyg är
\emph{IDA Pro} från Hex rays~\cite{hex_rays} och \emph{Binary Ninja} från Vector
35~\cite{binary_ninja}.
