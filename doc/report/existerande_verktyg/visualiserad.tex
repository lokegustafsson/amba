De i avsnitt~\ref{sec:existerande-disasm},\ \ref{sec:existerande-ramverk} samt\
\ref{sec:existerande-automatisk} presenterade kategorierna av binäranalysverktyg
är alla etablerade kategorier som AMBA inte faller under. Kategorin som AMBA
tillhör kallar vi ''visualiserad symbolisk fuzzing'' och detta avsnitt
diskuterar det fåtal existerande verktyg vi anser faller inom kategorin.

Verktyget Symbolic Execution Debugger
(\textsc{SED})~\cite{symbolic_execution_debugger} är en debugger för Java 1.4
utan multitrådning, reflektion och flyttalsaritmetik. Användaren använder
\textsc{SED} som ett Eclipse-plugin, i vilket valfri metod kan väljas och
analyseras med symbolisk exekvering. Exekveringsträdet visas för användaren
tillsammans med källkodsrader och de vägvillkor som tar exekveringen dit.

Att \textsc{SED} är utformat som en debugger som kan starta exekveringen vid
valfri Javametod gör verktyget väl utformat till formell verifiering. Verktyget
förlitar sig på tillgång till källkod och att användaren vet vilken indata som
anses giltig för metoden. Begränsningen till enskilda metoder gör samtidigt den
symboliska exekveringen mer överblicklig för användaren, då användaren i mindre
grad själv behöver exkludera det som analysen inte bör lägga fokus på.

Att \textsc{SED} inriktar sig på en delmängd av Java innebär att det är mindre
applicerbart för datasäkerhet och binäranalys då dessa sammanhang ofta kräver
analys av exekverbara binärer, ofta utan tillgång till källkod. Eftersom Java är
ett språk på högre nivå än maskinkod innebär samtidigt en inriktning mot Java
att många implementationsdetaljer i hur koden skulle göras på en riktig dator
kan ignoreras. Dynamisk minneshantering är ett exempel på detta, där analys av
Javakod kan abstrahera bort exekveringsmiljöns anrop till \verb|malloc()| och
\verb|free()| medan analys av maskinkod nödvändigtvis behöver hantera
komplexiteten i deras implementationer. Symbolisk exekvering i en högnivåmiljö
exkluderar därmed också många detaljer som verktyget eller användaren annars på
annat vis hade behövt filtrera bort.

Verktyget SymNav~\cite{symnav} kan visualisera och kontrollera analysen av en
exekverbar binär inuti angr:s symboliska exekveringsmotor.\@ angr beskrivs i
avsnitt~\ref{sec:existerande-ramverk}.

Arbetsflödet består av att starta SymNav med en via kommandorad angiven binär
och sedan vänta tills programmet forkar. Då visar det grafiska gränssnittet det
symboliska trädet, delar av kontrollflödesgrafen, etc. Användaren kan då
navigera den visualiserade informationen, och sedan antingen fortsätta eller
starta om körningen med en angiven tidsbudget och minnesbudget. Till exempel kan
användaren starta en sökning med standardparametrar i $10$ sekunder och med
$\SI{1}{\giga\byte}$ RAM.\@ När söktiden löpt ut uppdateras informationen i
det grafiska gränssnittet.

SymNavs grafiska gränssnitt består av tre paneler: en trädpanel som visar trädet
av symboliska tillstånd, en kontrollflödespanel som visar kontrollflödesgrafen
för enskilda funktion och en styrpanel för att filtrera vilken information som
ska visas.

Trädpanelen visar en kompakt representation av alla vägar av symboliska
tillstånd som tagits. Dessutom visas en parallellkoordinatgraf där diverse
nyckeltal för de symboliska tillstånden visas. Genom att interagera med denna
graf kan användaren filtrera bort vägar efter nyckeltal, till exempel genom att
endast visa vägar som någon gång anropar \verb|malloc()|.

Kontrollflödespanelen visar alla en graf över grundblock inom en funktion. Noder
innehåller den assemblykod som deras grundblock innehåller och kanter färgläggs
efter hur många olika symboliska tillstånd som vandrat kanten.

Styrpanelen visar för användaren vilka filter de hittills applicerat och låter
användaren filtrera ytterligare vilka symboliska tillstånd som ska visas i det
grafiska gränssnittet. Dessutom låter styrpanelen användaren definera
resursbudgeten, starta om körningen eller fortsätta körningen från nuvarande
symboliska tillstånd. När en körning startas kan användaren ange att nuvarande
filter i det grafiska gränssnittet ska omvandlas till villkor på den symboliska
indatan samt prioriteringsregler för fortsatt sökning.

SymNav är därmed fasbaserad, med alternerande söknings- och presentationsfaser.
SymNav ger användaren kraftfulla verktyg att filtrera ut intressanta tillstånd
efter metriker såsom kodtäckningsgrad och minnesallokeringsmönster. Samtidigt
ärver SymNav flera nackdelar från angr, såsom suboptimal prestanda
\cite{systematic_comparison_symbex} och begränsad miljömodellering då
analysgränsen går vid processen med emulerade syscalls snarare än vid en
virtuell maskin. SymNav är skrivet för att kunna anpassas till andra
exekveringsmotorer än angr, men detta är inte implementerat i dess nuvarande
form.
