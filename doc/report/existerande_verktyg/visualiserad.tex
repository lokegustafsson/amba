De i avsnitt~\ref{sec:existerande-disasm},\ \ref{sec:existerande-ramverk} samt\
\ref{sec:existerande-automatisk} presenterade kategorierna av binäranalysverktyg
är alla etablerade kategorier som AMBA inte faller under. Kategorin som AMBA
tillhör kallar vi ''visualiserad symbolisk fuzzing'' och detta avsnitt
diskuterar det fåtal existerande verktyg vi anser faller inom kategorin.

Verktyget Symbolic Execution Debugger
(\textsc{SED})~\cite{symbolic_execution_debugger} är en debugger för Java 1.4
utan multitrådning, reflektion och flyttalsaritmetik. Användaren använder
\textsc{SED} som ett Eclipse-plugin, i vilket valfri metod kan väljas och
analyseras med symbolisk exekvering. Exekveringsträdet visas för användaren
tillsammans med källkodsrader och de vägvillkor som tar exekveringen dit.

Att \textsc{SED} är utformat som en debugger som kan starta exekveringen vid
valfri Javametod gör verktyget väl utformat till formell verifiering. Verktyget
förlitar sig på tillgång till källkod och att användaren vet vilken indata som
anses giltig för metoden. Begränsningen till enskilda metoder gör samtidigt den
symboliska exekveringen mer överblicklig för användaren, då användaren i mindre
grad själv behöver exkludera det som analysen inte bör lägga fokus på.

Att \textsc{SED} inriktar sig på en delmängd av Java innebär att det är mindre
applicerbart för datasäkerhet och binäranalys då dessa sammanhang ofta kräver
analys av exekverbara binärer, ofta utan tillgång till källkod. Eftersom Java är
ett språk på högre nivå än maskinkod innebär samtidigt en inriktning mot Java
att många implementationsdetaljer i hur koden skulle göras på en riktig dator
kan ignoreras. Dynamisk minneshantering är ett exempel på detta, där analys av
Javakod kan abstrahera bort exekveringsmiljöns anrop till \verb|malloc()| och
\verb|free()| medan analys av maskinkod nödvändigtvis behöver hantera
komplexiteten i deras implementationer. Symbolisk exekvering i en högnivåmiljö
exkluderar därmed också många detaljer som verktyget eller användaren annars på
annat vis hade behövt filtrera bort.

Verktyget SymNav~\cite{symnav} kan visualisera och kontrollera analysen av en
exekverbar binär inuti angr. angr beskrivs ingående i
avsnitt~\cite{sec:existerande-ramverk}.
