Det kan finnas flera metoder för analys av binärprogram och det finns ganska
många verktyg idag som stödjer ett eller flera av dessa metoder. Det finns
många tillgängliga binäranalysverktyg men några populära binäranalysverktyg är
Ghidra~\cite{ghidra_website} och Angr~\cite{angr_web}.

% Sure I am referencing to the Ghidra website, but this is done by others as
% well because Ghidra has no report afaik. An example report that uses ghidra
% https://rp.os3.nl/2019-2020/p49/report.pdf

% Vidare har angr t.ex. sagt att "om man använder angr som en komponent i sin
% projekt ska man helst citera en viss rapport", men vi beskriver bara med
% några ord vad angr är och använder den till inget.

Angr är en binäranalysverktyg med stöd för både statiska och dynamiska analyser
med hjälp av symbolisk exekvering. Angr har, som Ghidra, stöd för
disassemblering till pseudo-C-kod och många analyser man kan utföra. Angr är
baserat på en emulator skriven i Python med stöd för symbolisk exekvering och
analyser utförs genom Python-skript som interagerar med Angrs API. Angr har
använts för att framställa skript som kan utföra \emph{reverse engineering},
sårbarhetssökning och fungera som exploateringsverktyg~\cite{angr_docs}.
