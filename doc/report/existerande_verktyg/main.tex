All form av binäranalys kan delas in i statisk; dynamisk; eller en kombination
av dem~\cite{andriesse2018}. Statiska metoder bygger på att man översätter
maskinkod till assembler utan att att exekvera binären, medan dynamiska
metoder loggar exekverade instuktioner~\cite{andriesse2018}. Dessa analysmetoder
implementeras av en rad olika verktyg som idag används inom akademin och
industrin. Listans syfte är att ge läsaren en översikt över fältet, och målet är
inte att vara komplett.

% Sure I am referencing to the Ghidra website, but this is done by others as
% well because Ghidra has no report afaik. An example report that uses ghidra
% https://rp.os3.nl/2019-2020/p49/report.pdf

% Vidare har angr t.ex. sagt att "om man använder angr som en komponent i sin
% projekt ska man helst citera en viss rapport", men vi beskriver bara med
% några ord vad angr är och använder den till inget.

Angr är en binäranalysverktyg med stöd för både statiska och dynamiska analyser
med hjälp av symbolisk exekvering. Angr har, som Ghidra, stöd för
disassemblering till pseudo-C-kod och många analyser man kan utföra. Angr är
baserat på en emulator skriven i Python med stöd för symbolisk exekvering och
analyser utförs genom Python-skript som interagerar med Angrs API. Angr har
använts för att framställa skript som kan utföra \emph{reverse engineering},
sårbarhetssökning och fungera som exploateringsverktyg~\cite{angr_docs}.
