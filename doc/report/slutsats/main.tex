Syftet med arbetet var att utveckla ett binäranalysverktyg för att interaktivt
visualisera symbolisk fuzztestning. Visualiseringen skulle presenteras i ett
grafiskt gränssnitt där användaren kan integrera med den virtulla maskinen som
exekverar binären som analyseras.

Ett av projektets mål var att med visuella representationer av exekveringsstigar
göra det enklare för binäranalytiken att förstå binären. Med möjligheten att
ändra visualiseringen binärs kontrollflöde mellan tre olika grafer, och därmed
tre olika perspektiv.

AMBA kan även göra statiska analyser av maskinkod, vilket förenklar för
användaren då användaren inte behöver använda líka många självständiga
analysprogram.

Att bygga AMBAs beroenden visades vara mycket svårt, vilket medförde att
slutprodukten fick lida på bekostnad av projektets byggsystem. Medan beroendrna
medför en större teoretisk potential var konsekvenserna av beroenderar
koplexitet att den praktiska potentialen blev mindre.

Våra förhoppningar är att AMBA antingen kommer att vidareutvecklas, eller ligga
till grund för ett nytt verktyg, som möjliggör state merging; övervakning av
systemanrop; samt sparande av SMT-lösarens resultat. Med den förbättrade
bygginfrastrukturen och grunden för ett symboliskt fuzzingverktyg hoppas vi att
utvecklingen av denna typ av system är både enklare och simplare.
