I avsnitt~\ref{sec:mal} beskriver vi projektets mål som att utveckla ett
binäranalysverktyg för att interaktivt visualisera symbolisk fuzztestning.
Visualiseringen presenteras i ett grafiskt gränssnitt där användaren kan
integrera med den virtulla maskinen som exekverar binären som analyseras.

Ett av projektets mål var att med visuella representationer av exekveringsstigar
göra det enklare för binäranalytikern att förstå binären. Med möjligheten att
ändra visualiseringen av en binärs kontrollflöde mellan fem olika grafer, och
därmed fem olika perspektiv. Idag uppnår AMBA det här målet under väldigt
begränsade förhållanden. Under arbetets gång observerade di att skapa ett
generellt visualiseringsverktyg av symbolisk fuzzingtestning är för omfattande
för att hinnas med under ett kandidatarbete. Med vidareutveckling, som
diskuteras mer i detalj i avsnitt~\ref{sec:vidareutveckling}, tror vi att AMBA
kommer vara praktiskt användbar under mer generella förhållanden.

Som diskuterat i avsnitt~\ref{sec:arbetsprocess} var ett av våra mål att
paketera AMBA tillsammans med dess komplexa beroenden. Att bygga AMBAs beroenden
visades vara mycket svårt då \stoe{}s bygginstruktioner är utdaterade och
fungerar inte längre. Till exempel hade \stoe{} projektets kopia av QEMU
flyttats, men länken hade inte uppdaterats i git vilket ledde till att kopian
behövde lokaliseras manuellt. Som nämnts i avsnitt \ref{sec:arbetsprocess} lades
även en ansenlig mängd tid på att försöka bygga guest images som skulle köras
inuti QEMU, men då vi inte lyckade producera samtliga artefakter beslutade vi
att använda de färdigbyggda guest images distribuerade av \stoe{} från Google
Drive. Då mycket tid behövde dedikeras till att korrigera byggsystemet innebar
det att mindre tid kunde allokeras till utveclandet av slutprodukten.

Vi försökte även minimera mängden \smallcaps{C++} kod som behövde skrivas genom
att använda Autocxx för att generera kopplingar mellan \smallcaps{C++} och Rust.
Tanken var att med hjälp av Autocxx skulle det endast ett fåtal rader
\smallcaps{C++}, och att merparten av källkoden skulle skrivas i Rust. Tyvärr
stötte vi på en bug i \smallcaps{Autocxx}, då den bland annat genererade
syntaktiskt inkorrekt kod. Då vi upptäckte att \smallcaps{Autocxx} saknade
viktiga funktioner beslutade vi oss efter några veckor att överge tanken att
generera kopplingar, för att skriva en större mängd \smallcaps{C++} kod.

Våra förhoppningar är att AMBA antingen kommer att vidareutvecklas, eller ligga
till grund för ett nytt verktyg, med funktionerna som diskuterades i avsnitt
\ref{sec:vidareutveckling}. För närvarande anser vi inte att AMBA är praktiskt
användbart, men att det finns en stor potential vid vidareutvecklingen av AMBA.
