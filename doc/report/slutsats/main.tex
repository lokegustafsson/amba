Syftet med arbetet var att utveckla ett binäranalysverktyg för att interaktivt
visualisera symbolisk fuzztestning. Visualiseringen skulle presenteras i ett
grafiskt gränssnitt där användaren kan integrera med den virtulla maskinen som
exekverar binären som analyseras.

Ett av projektets mål var att med visuella representationer av exekveringsstigar
göra det enklare för binäranalytiken att förstå binären. Med möjligheten att
ändra visualiseringen binärs kontrollflöde mellan tre olika grafer, och därmed
tre olika perspektiv. Idag uppnår AMBA det här målet under väldigt begränsade
förhållanden. Under arbetets gång observerades att skapa ett generellt
visualiseringsverktyg av symbolisk fuzzingtestning är för omfattande för att
hinnas med under ett kandidatarbete. Med vidareutveckling, som diskuteras mer
i detalj i avsnitt \ref{sec:vidareutveckling}, tror vi att AMBA kommer vara
praktiskt användbar under mer generella förhållanden.

Som diskuterat i avsnitt \ref{sec:arbetsprocess} var ett av våra mål att
paketera AMBA tillsammans med dess komplexa beroenden. Att bygga AMBAs beroenden
visades vara mycket svårt då \stoe{}s bygginstruktioner är utdaterade och
fungerar inte längre. Till exempel hade \stoe{} projektets kopia av QEMU
flyttats, men länken hade inte uppdaterats i git vilket ledde till att kopian
behövde manuellt lokaliseras. Som nämnts i avsnitt \ref{sec:arbetsprocess} lades
även en ansenlig mängd tid på att försöka bygga operativsystemet Ubuntu som
körs i QEMU, men då ingen lösning hittades beslutade vi att använda färdigbyggda
kopior distribuerade av \stoe{} från Google Drive. Då mycket tid behövde
dedikeras till att korrigera byggsystemet innebar det att mindre tid kunde
allokeras till utveclandet av slutprodukten.

Våra förhoppningar är att AMBA antingen kommer att vidareutvecklas, eller ligga
till grund för ett nytt verktyg, med funktionerna som diskuterades i avsnitt
\ref{sec:vidareutveckling}. För närvarande anser vi inte att AMBA är praktiskt
användbart, men att det finns en stor potential vid vidarutvecklingen av AMBA.