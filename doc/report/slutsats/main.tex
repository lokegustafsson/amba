I avsnitt~\ref{sec:mal} beskrives projektets mål om att utveckla ett
binäranalysverktyg för att interaktivt visualisera symbolisk fuzzing.
Visualiseringen skulle presenteras i ett grafiskt gränssnitt där användaren kan
interagera med den symboliska fuzzingen av det analyserade programmet. Detta mål
har uppnåtts i och med att AMBA implementerats med nuvarande funktionalitet.

Projektet motiverades av hypotesen att visuella representationer av
exekveringsvägar gör det enklare för binäranalytikern att förstå binären,
jämfört med ett skriptdrivet användargränssnitt som är standard vid direkt
användning av \stoe{}, angr, SymQEMU och liknande. Vi anser att AMBA idag uppnår
stödjer denna hypotesen under väldigt begränsade förhållanden. För att
vetenskapligt undersöka hypotesen är det rimligt att först utöka AMBA eller ett
liknande program såsom SymNav med funktionalitet som gör det praktiskt
användbart, och/eller därefter genomföra en empirisk användarvänlighetsstudie.
Att genomföra en sådan var inte ett mål och har inte heller genomförts i detta
kandidatarbete.

Som nämnt i avsnitt~\ref{sec:amba-vs-symnav} anser vi inte att AMBA är praktiskt
användbart. Men som vi nämnt i avsnitt~\ref{sec:vidareutveckling} anser vi att
AMBA har möjlighet att bli användbart och fylla en unik nisch, om ytterligare
funktionalitet implementeras.

Som diskuterat i avsnitt~\ref{sec:arbetsprocess} var ett av våra mål att
paketera AMBA tillsammans med dess komplexa beroenden. Att bygga AMBAs beroenden
visade sig vara mycket svårt då \stoe{}s bygginstruktioner är ofullständiga och
delvis utdaterade. Att mycket tid behövde dedikeras till att korrigera
byggsystemet innebar att mindre tid kunde allokeras till utvecklandet av
slutprodukten.

Våra förhoppningar är att AMBA antingen kommer att vidareutvecklas, eller
inspirerar till ett nytt verktyg, med funktionerna som diskuterats som
vidareutveckling. För närvarande anser vi inte att AMBA är praktiskt användbart,
men anser att det finns en stor potential i vidareutveckling av AMBA.\@
