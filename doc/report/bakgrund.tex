% BAKGRUNDENS UPPGIFT?: Motivera behovet att visualiseringsverktyg för fuzzing
% (fuzzing används slarvigt, inkludera tex concolic testing)
\subsection{Bakgrund}

Problematisk minneshantering har potential att påverka ett programs korrekthet och 
kan utnyttjas av fientliga aktörer i skadliga syften. Att minne hanteras på ett 
osäkert sätt är inte ovanligt, speciellt då proggrammet är skrivet i ett språk som är 
"memory unsafe" som exempelvis C/C++. Det är då lätt att vid utveckling av program 
göra misstag som introducerar sårbarheter, och kan vara svårare att upptäcka dessa 
sårbarheter när de väl introducerats, speciellt om det inkorrekta beteendet endast 
uppstår under körning med specifika indata.

Att läsa källkod är ett sätt att förstå program, men ibland är det gynnsamt att istället betrakta
maskinkoden direkt. Detta kan vara för att
\begin{itemize}
  \item utesluta påverkan av kompilatorbuggar som ger oväntad maskinkod
  \item se hur UB (\emph{undefined behaviour}) har utnyttjats av kompilatorn
  \item källkoden inte är tillgänglig
  \item i ett rekreationellt syfte köra Capture the Flag
\end{itemize}

Det finns en uppsjö av metoder som kan användas för att analysera en exekverbar
binär. Exempel på dessa är att: 
\begin{enumerate}
  \item disassemblera binären och läsa dess funktioner för att förstå vad de gör.
  \item dekompilera assemblykoden med ett verktyg som ger pseudokod, och sedan läsa denna mer
    läsbara koden.
  \item köra binären på speciella testfall och jämföra svaret med vad som förväntas. Om
    programmet implementerar en specifikation kan en existerande testsamling användas.
  \item fuzztesta binären, det vill säga automatiskt generera testfall tills ett orsakar en crash eller
    annat oönskat beteende i binären. Många fuzztestmotorer skapar testfall med en evolutionär
    algoritm, och många använder instrumentering över vilka programhopp som tas för att bedöma
    testfalls nyttighet.
  \item använda concolic testing, alltså fuzzing där en SMT solver genererar nya testfall genom att
    lösa för testfall som orsakar annorlunda programhopp.
  \item stega igenom programmet i en debugger för att se exakt vad programmet gör med viss input.
\end{enumerate}

Samtliga listade metoder är även exempel på metoder som används i reverse
engineering-sammanhang, något som är centralt i denna rapport.

Begreppet \textit{reverse engingeering} syftar på processen att söka insikt i hur en produkt 
(enhet/process/mjukvara/verktyg/system) arbetar, utan en etablerad insikt i dess interna 
uppbyggnad. Med andra ord syftar reverse engineering på att dekonstruera en produkt för att 
öka förståelsen av den. Detta görs genom att med olika metoder plocka isär produkten för 
att förstå hur den utför ett arbete. Reverse engineering är ett fundamentalt verktyg då insikt 
om en produkts design behövs men designspecifikationer ej existerar eller är tillgängliga. 
Reverse engineering har flera användningsområden, däribland då äldre produkter, vars design 
inte längre är tillgänglig, behöver undersökas, eller när funktionalitet försvunnit i 
utvecklingsaproccesen och behöver återfinnas. Reverse engineering är också användbart för 
att analysera fel som uppstår, för att förbättra delkomponenter eller för att diagnostisera 
en produkt.

För att bilda en allmän förståelse om ett program krävs både \textit{korrekt} och
\textit{abstrakt} förståelse. I detta avseende syftar \textit{korrekt} på
avsaknaden av felaktiga slutsatser och \textit{abstrakt} på möjligheten att
resonera om programmet generellt i motsats till att resonera om en specifik
konkret indata i taget.

Metod 1-2, att läsa kod, kan ge en \textit{abstrakt} förståelse av vad
programmet gör, men för att verifiera att huruvida resonemanget är korrekt krävs
hypotestestning vilket kräver att programmet körs. Således går det inte att
bilda en \textit{korrekt} förståelse genom att enbart läsa kod.

Metod 3-5, att köra programmet på testfall, ger framförallt en
black-box-förståelse av programmet. Tillgången till binären och
exekveringsmiljön används endast som ett verktyg för att generera nya testfall.
Fuzzing och concolic testing kan köras helautomatiskt och är \textit{korrekta}.
Men ofta är en tillräckligt täckande sökning av indatarummet omöjlig, och då kan
den automatiska analysen ha missat ett kvalitativt annorlunda beteende. Dessutom
ger en omfattande uppsättning indata-utdata-par inte användaren samma
information som källkoden ger. Därmed är helautomatiska analysmetoder inte
\textit{abstrakta}. Notera att det inte nödvändigtvis tyder på en brist i den
automatiska analysen att ett kvalitativt annorlunda beteende missas, för det
gömda beteendet skulle kunna vara en konsekvens av komplicerad kod, som till
exempel ett hoppvilkor beroende på en kryptografisk hash av indatan. Men en
analysmetod borde kunna peka ut var dess förståelse tar slut, snarare än att
utelämna detta fullständigt vilket är vad avsaknaden av testfall visar sig som.

Med metod 6, en debugger, kan användaren följa exekveringen för en viss indata
utan att riskera att missförstå hur datan transformeras. Om användaren har ett
oändligt tålamod kan de göra detta om och om igen för olika indata genererade
med till exempel fuzzing. Varje genomstegning ger information om koden som
behandlar indatan men också viss information om övrig kod -- till exempel kan
ett svårtaget hopp indikera en plats för användaren att rikta sin uppmärksamhet
mot. Detta ger en både \textit{korrekt} och \textit{abstrakt} förståelse, men
med en orimlig manuell arbetsbörda för användaren.

En helautomatisk \textit{korrekt} metod kan ge en \textit{abstrakt} förståelse
om processens förlopp visualiseras för användaren. Valet mellan manuell
arbetsbörda som ger djup förståelse och en testfallsgenerationsdriven process
som ger översiktlig förståelse kan genomföras av användaren om verktygen stödjer
hela spektrummet.

\subsubsection{Statisk och dynamisk binäranalys}
En annan typ av kategorisering av olika analysmetoder som fokuserar på hur
analysen genomförs delar metoderna i två grupper: statisk och dynamisk analys
\cite{dynamic_bin_analysis}.

Statisk analys syftar på analys som går att göra utan att exekvera programmet
som analyseras. Exempel på statisk binäranalys är metod 1-2, alltså att
disassembla binären och/eller visualisera kod.

Dynamisk analys går ut på att analysera ett program under exekvering. Exempel
på dynamisk binäranalys är metod 3-6. I alla fall krävs någon typ av injektion
av kod eller data i programmet utan att ändra på dess beteeende i syfte att
kunna extrahera viktig information under programmets körning. Många avancerade
dynamiska metoder som t.ex. concolic testing kräver symbolisk exekvering. Det
går ut på att tilldela symboliska värden till variabler och se hur dessa
påverkar programhopp och förgreningar och vad för möjliga värden som denna
symbol kan inneha under exekvering. I fallet med concolic testing används denna
information för att, med hjälp av en SMT-solver ta fram konkreta värden som
leder till att programmet kör till en program-distination som består av ett
oväntad krasch.

\subsubsection{Existerande verktyg}
Det kan finnas flera metoder för analys av binärprogram och det finns ganska
många verktyg idag som stödjer ett eller flera av dessa metoder. Det är inte
möjligt att lista och gå igenom alla tillgängliga binäranalysverktyg men de
mest populära exemplen är Ghidra\cite{ghidra_website} och
Angr\cite{angr_github}.

Ghidra är en \emph{reverse engingeering} ramverk utvecklat av USA:s NSA
(National Security Agency) och kan disassembla en binär till pseudo-C-kod.
Ghidra har också en debugger och funktionsgraf som ska underlätta binär
debugging genom att integrera med andra funktioner i Ghidra och funktionsgrafen
låter användaren se hur programmet är uppbyggt visuellt och hur olika
funktioner interagerar med varandra. Funktionaliteter kan utökas eller andra
funktioner utvecklas genom plugins till Ghidra\cite{ghidra_use_cases}. Ghidra
tillåter även automatisering genom att skriva skript. Ett exempel är ett skript
som hittar exempelvis sårbarhet i form av funktionsanropp till potentiellt
osäkra API-anrop genom statisk analys\cite{ghidra_script}.

Angr är en binäranalysverktyg med stöd för både statiska och dynamiska analyser
med hjälp av symbolisk exekvering. Angr har, som Ghidra, stöd för
disassemblering till pseudo-C-kod och många analyser man kan utföra. Angr är
baserat på en emulator skriven i Python med stöd för symbolisk exekvering och
analyser utförs genom Python-skript som interagerar med Angrs API. Angr har
använts för att framställa skript som kan utföra \emph{reverse engineering},
sårbarhetssökning och fungera som exploateringsverktyg\cite{angr_docs}.

