\chapter{Inledning}

Attacker mot organisationers eller nationers IT-system har ofta förrödande
direkta och indirekta konsekvenser för de drabbade. Enligt företaget Cybereason
svarade över 70 procent av de över 1400 tillfrågade säkerhetsspecialister i en
av deras studier att deras organisation hade utsats för
\emph{Utpressningsprogram} (engelska: \emph{ransomware}) minst en gång de senaste
24 månaderna \cite{cyberreason2021, cyberreason2022}. Cybereasons rapporter
beskriver att några av de direkta följderna av att en lyckad attack är höga
kostnader för organisationen; stoppad produktion; konsekvenser från reglerare;
och skadat anseende. Enligt Cybereasons rapporter leder de höga kostnaderna till
bland annat uppsägningar av anställda för att kunna kompensera. I en artikel i
Bloomberg Newsweek \cite{gallagher2023} beskrivs hur cancerpatienter som behövde
akut sjukvård förflyttades till sjukhus med fungerade datorsystem när Irlands
offentliga sjukhussystem låg nere. \cite{hse_report} Ransomware attacker
möjliggörs tack vare säkerhetsbrister i programvarorna som finns på offrens
datorsystem.

Under \emph{BlueHat} 2019 gav \emph{MSRC} (\emph{Microsoft Security Response Center}) en överblick
över Microsofts taktiker för hur de hanterar säkerhetsbrister i deras produkter och tjänster.
\cite{miller19} Taktikerna som listades var att
\begin{itemize}
	\item eliminera säkerhetsbristerna från början;
	\item implementera metoder för att försvåra utnyttjande av säkerhetsbrister;
	\item minimera platserna där attackerarna kan göra skada och förhindra åtkomst; samt slutligen
	\item minimera tidsfönstret då attackerare har tillgång till systemet med hjälp av aktiv övervakning.
\end{itemize}
Enligt Miller \cite{miller19} härstammar ungefär 70 procent av alla Microsofts CVEr (Common Vulnerabilities and Exposures)https://en.wikipedia.org/wiki/Sci-Hubzz från minnessäkerhetsbuggar. Om
hela klasser av säkerhetsbrister kan eliminieras genom att utveckla nya verktyg som underlättar för
utvecklare att finna dem kan det leda till stor påverkan på antalet sårbarheter.zyyy

% probably need a source for this paragraph
% explain machine code?
Det är särskilt fördelaktigt att undersöka minnessäkerhetsbuggar genom att
betrakta maskinkod. Genom att betrakta maskinkod bildar man en lågnivåförståelse
av en binär eftersom den fullständiga målkoden som kompilatorn genererat
existerar på denna nivån. Med en lågnivåförståelse kan man också härleda hur
binären interagerar med minnet genom att bland annat analysera minneslayouten
och den underliggande datastrukturen för att hitta problem vid minnesallokering
och minnesdeallokering, och därigenom hitta minnessäkerhetsbuggar.

Det finns ett flertal metoder för att analysera en exekverbar binär. Exempel på dessa är att: 
\begin{enumerate}
  \item disassemblera binären och läsa dess funktioner för att förstå vad de gör.
  \item dekompilera assemblykoden med ett verktyg som ger pseudokod, och sedan läsa denna mer
    läsbara koden.
  \item köra binären på speciella testfall och jämföra svaret med vad som förväntas. Om
    programmet implementerar en specifikation kan en existerande testsamling användas.
  \item fuzztesta binären, det vill säga automatiskt generera testfall tills ett orsakar en crash eller
    annat oönskat beteende i binären. Många fuzztestmotorer skapar testfall med en evolutionär
    algoritm, och många använder instrumentering över vilka programhopp som tas för att bedöma
    testfalls nyttighet.
  \item använda concolic testing, alltså fuzzing där en SMT solver genererar nya testfall genom att
    lösa för testfall som orsakar annorlunda programhopp.
  \item stega igenom programmet i en debugger för att se exakt vad programmet gör med viss input.
\end{enumerate}

Problematisk minneshantering har potential att påverka ett programs korrekthet och 
kan utnyttjas av fientliga aktörer i skadliga syften. Att minne hanteras på ett 
osäkert sätt är inte ovanligt, speciellt då proggrammet är skrivet i ett språk som är 
"memory unsafe" som exempelvis C/C++. Det är då lätt att vid utveckling av program 
göra misstag som introducerar sårbarheter, och kan vara svårare att upptäcka dessa 
sårbarheter när de väl introducerats, speciellt om det inkorrekta beteendet endast 
uppstår under körning med specifika indata.

