\section{Inledning}

Attacker mot organisationers eller nationers IT-system har ofta förrödande
direkta och indirekta konsekvenser för de drabbade. Enligt företaget Cybereason
svarade över 70 procent av de över 1400 tillfrågade säkerhetsspecialister i en
av deras studier att deras organisation hade utsats för
\emph{Utpressningsprogram} (engelska: \emph{ransomware}) minst en gång de senaste
24 månaderna \cite{cyberreason2021, cyberreason2022}. Cybereasons rapporter
beskriver att några av de direkta följderna av att en lyckad attack är höga
kostnader för organisationen; stoppad produktion; konsekvenser från reglerare;
och skadat anseende. Enligt Cybereasons rapporter leder de höga kostnaderna till
bland annat uppsägningar av anställda för att kunna kompensera. I en artikel i
Bloomberg Newsweek \cite{gallagher2023} beskrivs hur cancerpatienter som behövde
akut sjukvård förflyttades till sjukhus med fungerade datorsystem när Irlands
offentliga sjukhussystem låg nere. \cite{hse_report} Ransomware attacker
möjliggörs tack vare säkerhetsbrister i programvarorna som finns på offrens
datorsystem.


\subsection{Tidigare arbete}
Flertalet arbeten existerar inom domänen binäranalys och dess analystekniker. Fowze \cite{fowze_mem_vul}
beskriver verkyget \emph{SEESAW}, ett verktyg för att analysera
minnessårbarheter i protokollstacken. Fokuset ligger på att undersöka USB och
blåtandsmoduler med hjälp av en hybrid av analysmetoder: statisk analys och
dynamisk analys. \emph{SEESAW} implementerar en algoritm som tillämpar
bidirektionell kommunikation mellan en statisk analys som tillhandahåller ett
mål i binären till en riktad symbolisk exekvering dvs den dynamiska analysen som
ger tillbaka en bestämd funktionspekare för att komplettera och öka precisionen
av analysen.  

\subsection{Syfte}

\subsubsection{Mål}
Detta projekt ämnar utveckla ett binäranalysverktyg för generell reverse
engineering, det vill säga en applikation vars uppgift är att analysera binära
program utan kännedom om källkoden utifrån ett datorsäkerhetsperspektiv.
Verktyget ska ha en korrekt och testbar förståelse av det analyserade programmet
och ska samtidigt kunna kommunicera denna till användaren genom visualisering.
Visualiseringen ska presenteras i ett grafiskt användargränssnitt där användaren
kan interaktivt stega igenom det binära programmet och själv välja vilka beslut
som görs gällande exempelvis programhopp. Funktionaliteten i verktyget uppnås
med hjälp av symbolisk exekvering, och därigenom kombineras fördelarna i
automatiska och manuella analysmetoder.

% Utveckla syfte (förstå nyttan och det akademiskt relevanta i arbetet)
Verktyget ämnar att genom mänskligt läsbara representationer av programmets 
beteende, öka användarens abstrakta förståelse av det. Verktyget ska vara 
användbart för att öka användarens abstrakta förståelse av ett program då källkod 
saknas. Det vill säga verktyget ska hjälpa användaren att resonera om programmet 
generellt. 

\subsubsection{Användningsområden}
Att kunna avgöra ett programs beteende utifrån endast exkeverbar (binär) 
kod är viktigt när man undersöker potentiellt skadlig mjukvara eftersom
dess källkod oftast är okänd. För att upptäcka pågående, och motverka 
framtida attacker är det viktigt att förstå hur attackerna är utformade och
hur de beteer sig. 

Binär analys är också användbart då tredje-parts bibliotek används.
Att analysera säkerheten hos ett program kan behöva ske utan att involvera 
dess utgivare. Då är det användbart att kunna dra slutsatser om programmet 
endast ifrån dess maskinkod. 

Förutom att analysera program där källkoden inte är tillgänglig så är det även 
användbart att analysera ett programs maskinkod för att undersöka kompilatorbuggar.

%\section{akademisk relevans}

\subsection{Avgränsningar}

Att skapa en en exekveringsmotor med stöd för bland annat symbolisk exekvering
skulle i praktiken innebära implementering av en emulator. Det är en relativ
stor och tidskrävande uppgift som dessutom kräver nogrann testning för att vara
korrekt och tillförlitlig. För att fastställa korrekthet ska applikationen
därför utnyttja existerande verktyg för att exekvera programmet med stöd för
symbolisk exekvering. Detta möjliggör också bredare plattformssupport jämfört
med en hemmasnickrad emulator.


\subsection{\stoe}

\stoe\cite{s2e} är en plattform för symbolisk exekvering som bygger på QEMU:s
virtuella maskin och använder KLEE\cite{klee}, en motor för symbolisk
exekvering, som interpreter för att möjliggöra symbolisk exekvering. \stoe\ är i
sin tur utbyggbart med möjlighet för användaren att skriva ett eget plugin för
att utföra analyser och används inom säkerhetsforskning för att till exempel
analysera skadlig kod. \stoe\ användes som del av Galactica-systemet som spelade
i DARPA Cyber Grand Challenge\cite{s2e_website}. \stoe\ är öppen källkod,
väldokumenterat och underhålls aktivt.

\subsection{SymQEMU}

Ett alternativt verktyg för symbolisk exekvering är symQEMU\cite{symqemu},
som också kombinerar QEMU:s virtuella maskin med KLEE:s motor för symbolisk
exekvering. Till skillnad från \stoe\ kompilerar SymQEMU KLEE in i den
analyserade binären och har jämförelsevis hög prestanda. Däremot har SymQEMU
bristfällig dokumentation och är ej aktivt uppdaterat.

\subsection{Beslut}

Då SymQEMU ej uppdateras aktivt och har bristfällig dokumentation kommer \stoe\
användas. Projektet avgränsas i och med att existerande verktyg (\stoe) kommer
användas istället för att bygga en motor för symbolisk exekvering från grunden.

\subsection{Begränsningar}

Att använda \stoe\ innebär att arbetet avgränsas till att, i praktiken, skapa
ett plugin som avlyssnar och styr motorn. Varken emulator eller motor ska
byggas och de uppgifter som ingår i att skapa en exekveringsmotor exkluderas.

Det innebär att fokus flyttas ifrån motorns tekniska detaljer till resterande
jobbet med att utveckla en användbar slutprodukt som bygger ut \stoe:s redan
existerande funktionalitet med ett grafiskt användargränsnitt och möjlighet att
stega igenom, analysera och interaktivt besluta om värden under exekvering.

Beslutet innebär också att applikationens utformning blir bunden till \stoe:s
tekniska begränsningar.
