\section{Metod}

I det här avsnittet beskrivs strategi och tillvägagångssätt för utvecklingen
samt evaluering av projektets mål och hur det framställda verktyget kommer att
jämföras med existerande binäranalysverktyg.

\subsection{Val av strategi}

För att uppnå och redogöra för möjligheterna i en potentiell applikation
reflekterades det över två möjliga alternativ att fullfölja. Dels diskuterades
alternativet att bygga en symbolisk exekveringsmotor från grunden och fokusera
på dess tekniska detaljer och dels att använda en existerande produkt där
fokuset istället ligger på att bygga plugin vars syfte är att utöka den
existerande motorn med fler funktioner. I det senare fallet blir därmed
slutprodukten ett program som kommunicerar med ett plugin genom \stoe\ via IPC
eller liknande. 

Med syfte att göra framsteg valdes det att kolla vidare hur \stoe\ kan hjälpa i
frågan om att utöka funktionalitet i en existerande motor. \stoe\ är utvecklat
i programspråket C++ och med anledning av valet att utveckla i ett
annat programmeringsspråk, det vill säga Rust, krävs implementation av
C++-bindings. I ett första steg valdes det därför att se över hur detta går till
på lämpligt sätt och hur detta kan automatiseras med hjälp av andra verktyg
där bland annat autocxx ansågs som ett lovande alternativ.

\subsection{Tillvägagångssätt}

I avsikt att utveckla ett binäranalysverktyg, bestämdes det att ta fram demon
(demonstrationer) som ska leda det kontinuerliga utvecklingsarbetet.

Ett demo ska bestå av en analys med hjälp av verktyget som ökar förståelse för
ett exempelprogram. Med förståelse menas en abstrakt programförståelse
beskrivet i bakgrundsavsnittet. Exempelprogrammet ska framställas med en
egenskap som man skulle vilja kunna detektera med ett analysverktyg. Mer
konkret ska ett sådant exempelprogram inneha t.ex. minnessårbarhet, ett oväntad
krasch eller något annat sårbarhet som inte önskas i ett program.

Exempelprogrammen ska skrivas i C för enkelhetens skull eftersom C-programs
utseende i binärformat relativt väl motsvarar deras källkod. 

Sedan ska exempelprogrammet kompileras till en binär som analyseras med
analysverktyget med avsikt att detektera den kända egenskapen hos
exempelprogrammet. Om detta inte är möjligt med den nuvarande version av
analysverktyget ska analysverktyget utökas för att möjliggöra demot.

Detta tillvägagångssätt är önskvärt eftersom det tillåter att sätta uppnåbara
delmål som styr funktionalitet i analysverktyget som ska implementeras härnäst.
Dessutom kan flera demon utvecklas samtidigt vilket tillåter parallellism inom
utvecklingen och framsteg på flera fronter.

Tillvägagångssättet är jämförbart med den agila arbetsmetoden där utveckling sker 
evolutionärt. Detta är önskvärt då verktygets potentiella 
användningsbarhet inte är självklar på detaljnivå.

I ett senare och/eller parallellt steg ska ett grafiskt användargränssnitt
utvecklas som tillåter användaren att navigera programmet som analyseras och
göra egna beslut gällande förgreningar etc. där användaren själv bestämmer
interaktivt nästa beslut som görs. Även detta arbete styrs av demon som
framställs och funktionaliteter som önskas.

\subsection{Evaluering och jämförelse med andra verktyg}

För att det ska gå att testa verktygets förmåga och jämföra med andra
analysverktyg ska de demon som tas fram bestå av CGC-binärer som analyseras. I
det fall CGC-binären är för komplicerat ska den förenklas samtidigt som den
originala sårbarheten återfinns, eller delas in i flera demon, tills dess att
den kompletta CGC-binären kan analyseras och sårbarheten detekteras.

Samma CGC-binärer som kan analyseras med det framställda analysverktyget ska
analyseras med binäranalysverktygen \emph{Angr} och \emph{Ghidra} och
analysresultatet jämföras för att göra en kvalitativ evaluering av eventuella
för- och nackdelar med det framtagna analysverktyget jämfört med de ovan nämnda
existerande verktygen. Jämförelsen kommer att ta hänsyn till hur snabbt det
gick att genomföra analysen, hur mycket ansträngning det krävde från
användaren, om det överhuvudtaget gick att hitta sårbarheten i CGC-binären, och
eventuellt andra för och nackdelar.

Val av CGC-binärer att analysera kommer att vara slumpmässigt med undantag för
fall då det anses ta för långt tid att utveckla vårt verktyg för att tillåta
analys av programmet med hänsyn till tidsbegränsningen för detta projekt.
