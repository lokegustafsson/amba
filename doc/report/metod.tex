\subsection{Metod}

Det är inte självklart hur utvecklingen av verktyget ska gå till och vilken 
exakt funktionalitet applikationen ska förses med för att uppfylla syftet. I det här
avsnittet beskrivs strategi, tillvägagångssätt samt evaluering av syfte i
avseende för utvecklingen.

\subsubsection{Val av strategi}

För att uppnå och redogöra för möjligheterna i en potentiell applikation
reflekterades det över två möjliga alternativ att fullfölja. Dels diskuterades
alternativet att bygga en symbolisk exekveringsmotor från grunden och fokusera
på dess tekniska detaljer och dels att använda en existerande produkt där
fokuset istället ligger på att bygga plugin vars syfte är att utöka den
existerande motorn med fler funktioner. I det senare fallet blir därmed
slutprodukten ett program som kommunicerar med ett plugin genom \stoe\ via IPC
eller liknande. 

Med syfte att göra framsteg valdes det att kolla vidare hur \stoe\ kan hjälpa i
frågan om att utöka funktionalitet i en existerande motor. \stoe\ är utvecklat
i programspråket C++ och med anledning av valet att utveckla i ett
annat programmeringsspråk, det vill säga Rust, krävs implementation av
C++-bindings. I ett första steg valdes det därför att se över hur detta går till
på lämpligt sätt och hur detta kan automatiseras med hjälp av andra verktyg
där bland annat autocxx ansågs som ett lovande alternativ.

\subsubsection{Tillvägagångssätt}

I avsikt att uppnå syftet med projektet, mer specifikt att utveckla ett
binäranalysverktyg för generella problem, bestämdes det att ta fram prototyper som ska leda det
kontinuerliga utvecklingsarbetet.

En prototyp är ett exempelprogram skapt i syfte att undersöka vilken
kapacitet verktyget har för att sedan utöka dess funktionalitet. Ett
exempelprogram ska ha som mål att visa en förmåga i verktyget som underlättar
förståelse för, eller belyser en känd intressant egenskap. Ett typiskt
intressant exempel är program som gör många programhopp, innehåller säkerhetshål
eller liknande. För att detta tillvägagångssätt ska vara genomförbart ställs
kravet att prototyperna hålls enkla samtdiigt som de uppnår sina mål.
Fortsättningsvis ska prototypen skrivas i C eftersom C-programs utseende i
binärformat relativt väl motsvarar dess källkod. Detta förenklar binäranalys och
låter prototypen fokusera på sin huvudsakliga uppgift. 

Detta tillvägagångssätt är önskvärt eftersom det tillåter att sätta uppnåbara delmål som styr
funktionalitet som ska implementeras härnäst. Prototyperna fungerar som test på dessa
funktionaliteter och visar att dessa är korrekta. Dessutom kan flera prototyper 
utvecklas samtidigt vilket tillåter parallellism inom projektarbetet och
framsteg på flera fronter.

I ett senare och/eller parallellt steg ska ett grafiskt användargränssnitt
utvecklas som tillåter användaren att traversera genom applikationen och göra
egna beslut gällande förgreningar av programmet etc. där användaren själv
bestämmer interaktivt nästa beslut som görs. Även detta arbetet styrs av
prototyper och funktionaliteter som önskas.

\subsubsection{Evaluering av syfte}

För att vidare kunna bestämma huruvida en prototyp är lämplig och ska möjliggöras
genom utvidgning av verktyget kommer en avgränsning ske efter vad som är
rimligt.
Hur väl prototypens analysidé kan komma att tillämpas i verktyget och hur analysen
ökar användbarhet av verktyget kommer användas i detta avseende. Dessutom ska det sättas i perspektiv till den tid
som krävs för att göra prototypen genomförbar med tanke på tidsramen för projektet.

Eftersom det huvudsakligen är prototyperna som styr den slutgiltiga
funktionaliteten hos verktyget är det därför även dessa som styr hur väl syftet
uppnåtts. Ett konkret sätt att mäta detta är hur generellt det slutgiltiga
verktyget blir, det vill säga går det exempelvis att definiera nya tester, som verktyget
ska kunna hantera, utan att göra ytterligare förändringar i verktyget. Dessutom,
går det att använda verktyget för generell reverse engineering eller för att lösa övningar
inom CTF?

% Hur är detta kopplat till vårt syfte? Hur uppnår vi syftet med rapporten genom
% vald metod?

% (varför är typ besvarat i val av strategi-avsnittet)

% Hur besvarar vi huruvida valet av strategi är rimligt?

%
