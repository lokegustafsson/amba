En annan typ av kategorisering av olika analysmetoder som fokuserar på hur
analysen genomförs delar metoderna i två grupper: statisk och dynamisk
analys~\cite{dynamic_bin_analysis}.

Statisk analys syftar på analys som går att göra utan att exekvera programmet
som analyseras. Exempel på statisk binäranalys är metod 1--2, alltså att
disassembla binären och/eller visualisera kod~\cite{dynamic_bin_analysis}.

Dynamisk analys går ut på att analysera ett program under
exekvering~\cite{dynamic_bin_analysis}. Exempel på dynamisk binäranalys är
metod 3--6. I alla fall krävs någon typ av injektion av kod eller data i
programmet i syfte att kunna extrahera viktig information under programmets
körning~\cite{dynamic_bin_analysis}. Många avancerade dynamiska metoder som
t.ex.\ concolic testing kräver symbolisk exekvering som går ut på att tilldela
symboliska värden till variabler och se hur dessa påverkar programhopp och
förgreningar och vad för möjliga värden som denna symbol kan inneha under
exekvering. I fallet med concolic testing används denna information för att,
med hjälp av en SMT-solver ta fram konkreta värden som leder till att
programmet kör till en program-distination som består av ett krasch.
