
Kärnan i ett korrekt dynamisk binäranalysverktyg är en
\emph{exekveringsmotor}, en komponent som på ett korrekt vis kan köra
programmet. Att köra ett program innebär att ladda binären och dess bibliotek,
hoppa till startadressen och sedan köra enskilda instruktioner. Om
binäranalysverktyget ska kunna använda metoder som använder symbolisk exekvering
behöver denna exekvering av enskilda instruktioner också stödja symboliska
variabler. För att verktyget ska uppnå hög prestanda är det eftersträvansvärt
att de instruktioner som inte använder symboliska variabler utan endast agerar
på konkreta värden exekverar direkt på processorn som kör binäranalysverktyget.

Program i verkligheten kommunicerar på många sätt med sin omgivning. För
inbyggda system är denna omgivning fysisk och för program som kör ovanpå
operativsystem är denna omgivning en virtuell värld bestående av filer och IPC (inter
process communication). För att en \textit{exekveringsmotor} ska vara så
brett tillämpbar som möjligt behöver den också stödja flera sorters
omgivningskommunikation.

Sammantaget är kraven på en exekveringsmotor med stöd för symbolisk analys att den
\begin{enumerate}
    \item kan ladda en binär i en konkret miljö;
    \item kan introducera symboliska variabler i denna miljö genom till exempel kommunikation med omgivningen;
    \item kan exekvera konkreta delar av programmet med god prestanda genom att köra dessa delar
          avprogrammet som maskinkod; och
    \item kan exekvera symboliska delar av programmet på ett sätt som är tillräckligt kraftfullt för de
          symboliska analyser som exekveringsmotorn ska användas till.
\end{enumerate}

Dessutom behöver exekveringsmotorn kunna kontrolleras och dess
slutsatser kunna användas av och presenteras i resten av programmet.
Figur~\ref{schematic} visar förhållandet mellan användaren, analysverktyget och
dess exekveringsmotor.

\begin{figure}[H]
    \centering
    \begin{tikzpicture}

        \node [draw, fill=lime, minimum
            width=2cm, minimum height=1cm, ]  (user) {Användare};

        \node [draw, fill=yellow, minimum width=2cm, minimum height=1cm,
            right=2cm of user ] (tool) {Analysverktyg};

        \node [draw, fill=orange, minimum width=2cm, minimum height=1cm,
            right=2cm of tool ] (engine) {Exekveringsmotor};

        \draw[->, line width=.4mm] (user.-30) to[out=-30, in=-150]
        node[midway,below]{Beslut} (tool.-150);

        \draw[->, line width=.4mm] (tool.150) to[out=150, in=30]
        node[midway,above]{Visualisering} (user.30);

        \draw[->, line width=.4mm] (tool.-30) to[out=-30, in=-150]
        node[midway,below]{Kontroll} (engine.-150);

        \draw[->, line width=.4mm] (engine.150) to[out=150, in=30]
        node[midway,above]{Instrumentering} (tool.30);

    \end{tikzpicture}
    \caption{ Schematisk bild av ett binäranalysverktyg byggt kring en exekveringsmotor}\label{schematic}
\end{figure}

\subsection{Analyser}

Det finns många möjliga analyser som kan användas av ett binäranalysverktyg, där
\textit{analyser} avser en visualisering av en aspekt av det analyserade
programmets beteende eventuellt inklusive ett sätt för användaren att påverka
det analyserade programmets körning.

En konkret exekvering kan spåras och dess instruktionssekvens kan visas för
användaren med loopar identifierade. Flera exekveringar kan visualiseras på
samma sätt och deras instruktionssekvenser användas för att återskapa
kontrollflödesstrukturer som for- och while-loopar och if-satser.

Möjligheter för \textit{state merging} kan identifieras helautomatiskt eller av
användaren, alltså platser där flera exekveringar kan ersättas av en enda mer
generell exekvering som innehåller ursprungsexekveringarnas skillnader som
symboliska uttryck.

En uppsättning liknande exekveringar kan visas upp för användaren tillsammans
med information om när och hur de divergerar, för att till exempel avgöra när
olika delar av indatan används.

För analys av programmet är det också hjälpsamt för användaren att kunna stega
genom exekveringen steg för steg och ändra på värden för att ta de vägar de vill
analysera. Detta borde också kunna göras baklänges, det vill säga att användaren
väljer en slutdestination och låter programmet själv lista ut vilka värden som
behövdes läsas för att exekveringen skulle ta sig till den punkten.

En mer automatisk men ändå viktig funktionalitet är konstruktion av kod-täckande
indata. När testfall som besöker en uppsättning instruktioner är konstruerad kan
kraven på indatan över denna uppsättning programhopp analyseras och programmet
kan lista ut vilka aspekter av indatan som kan ändras utan att påverka
kodtäckningsgraden, och kommunicera detta till användaren.

Det finns många möjliga analyser. Användbarheten i ett verktyg mäts inte i
kvantiteten analyser utan i förmågan för de utvalda implementerade analyserna
att täcka användarens behov av förståelse av programmet.
