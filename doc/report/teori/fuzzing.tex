Fuzzing är i sig ett användbart automatiskt verktyg för att testa program efter
ofta svårupptäckta problem som minnesbuggar, krascher, etc. på grund av dess
enkelhet. Grundprincipen i fuzzing är att attackera en större mängd av möjlig
indata genom att generera oväntad, godtycklig eller felaktig data. Denna typ
av genererad indata resulterar ofta i syntaktiskt felaktig indata som inte kan
hanteras av målprogrammet. Således finns det anledning för utvecklingspotential,
något som lett till bland annat Mutation-Based Fuzzing och Generaton-based
fuzzing. Mutation-Based Fuzzing muterar känd giltig indata, t.ex. om strängen
"fuzz" är giltig indata kan detta muteras till "fuzzZZZZZ". Detta skiljer sig från
Generation-Based fuzzing, som genererar indata givet en modell för
domänen~\cite{fuzzing}.

\subsection{Problemet med fuzzing}
Fuzzing kräver ofta protokoll- eller domänkännedom för att kunna generera 

\subsection{Symbolisk fuzzing}



% skriv om code coverage (räkna kanter vs noder, vilket är bäst)
% KÄLLORRRRRRRRRr 🙃



