Fuzzing är en användbar automatiskt metod för att testa program i syfte att finna
svårupptäckta problem såsom minnesbuggar, krascher, etc.\ eftersom det kräver lite
manuellt arbete.
%Tydliggör varför det är enkelt: För att man ofta bara kan peka en fuzzer mot ett program och sedan låta det köra automatiskt. Man behöver inte göra särskilt mycket manuellt arbete anpassat till just den binären

Grundprincipen i fuzzing är att undersöka programmets beteende på mer varierad
indata genom att generera oväntad, godtycklig eller felaktig data. Denna typ av
genererad indata resulterar ofta i syntaktiskt eller semantiskt felaktig indata
som inte kan hanteras av målprogrammet. Det är däremot inte trivialt att generera
indata som täcker ett programs möjliga beteenden vilket har lett till bland annat
mutationsbaserad fuzzing (jfr.\ eng.\ mutation-based fuzzing) och genereringsbaserad
fuzzing (jfr.\ eng. generation-based fuzzing). Mutationsbaserad fuzzing muterar känd giltig indata,
t.ex.\ om strängen `fuzz' är giltig indata kan detta muteras till `fuzzZZZZZ'. Om
en användare exempelvis vill utföra fuzzing på bildhanteringsbiblioteket libjpeg skulle
detta innebära att skicka giltiga jpeg-bilder till fuzzern för att användas som
seeds, värden som används i pseudoslumptalsgeneratorer för att generera
pseudoslumptal, och sedan modifiera dessa. Detta skiljer sig från
genereringsbaserad fuzzing som genererar indata givet en modell för domänen ---
en fördel i jämförelse med mutationsbaserad fuzzing som kräver känd kvalitativ
indata~\cite{fuzzing}.

\subsection{Symbolisk fuzzing} Symbolisk fuzzing är en white-box fuzzingmetod
som nyttjar symbolisk exekvering för att maximera code coverage, det vill säga fuzzerns
förmåga att traversera över samtliga kanter och noder i programmets kontrollflödesgraf.
I motsats till grey-box-fuzzing så möjliggör symbolisk exekvering att en gren som inte tidigare tagits alltid väljs
och ökar således code coverage~\cite{challenges_fuzzing}. Som beskrivet i avsnittet \nameref{sec:symbolic_execution}
sker detta genom att emulera programmet och ersätta indata med symboliska representationer, vilka enklast liknas med
matematiska formler i form av algebraiska uttryck. Dessa uttryck byggs tillsammans med vägvillkor
och kan skickas till en SMT-lösare för att evalueras till konkreta värden.

\subsection{Problem med fuzzing} Ett problem är insikt om den underliggande
kodstrukturen. En viktig egenskap som fuzzing har och som används för att beskriva dess
effektivitet är code coverage. Black-box-fuzzing är exempel på en typ av fuzzing som
saknar vetskap om den underliggande kodstrukturen och endast genererar
slumpmässig indata, något som leder till ytlig testning av målprogrammet. I
kontrast till black-box-fuzzing finns det grey-box fuzzing, exempelvis
AFL~\cite{aflplusplus} som tillämpar binärinstrumentering, en
teknik för att observera eller manipulera en binär genom att modifiera källkoden
i binären. Genom binärinstrumentering fås information om underliggande basic blocks
som delger övergången till nästa basic block. Detta används sedan av AFL för att ge
feedback till fuzzing processen som använder information om code coverage
för en viss indata och repeterar denna process för att med den nya indatan försöka
maximera code coverage och därmed öka testytan~\cite{challenges_fuzzing}.

Fuzzers kräver ofta protokoll- eller domänkännedom för att kunna generera
indata. Detta är svårare med komplexa kodbaser eller bibliotek som saknar
trivial eller uppenbar indata och leder därmed till lägre code coverage.

White-box fuzzers är inte en allmän lösning till problemen med fuzzing, utan
har problem med exempelvis prestanda, vägexplosion, och falskt positiva
resultat. Det finns en stark korrelation mellan code coverage och bug
coverage~\cite{directed_greybox_fuzzing} men eftersom white-box fuzzing är ett
prestandakrävande verktyg kan detta leda till falskt positiva resultat genom att
resultatet från den symboliska exekveringen guidar fuzzern längs med en gren som
inte nödvändigtvis leder till en bugg, eller som är omöjlig. Trots den starka korrelationen
mellan code coverage och bug coverage, innebär detta inte att buggar kan uteslutas vid
testning med hög code coverage.

%Det har visat sig att enbart 3\% av Mozilla Firefox källkod innehåller sårbarheter~\cite{fault_prediction_vuln_pred}, och därför är det oproduktivt att blint använda code coverage som metod för att finna dessa.
