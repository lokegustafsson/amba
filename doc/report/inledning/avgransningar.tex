
Att skapa en en exekveringsmotor med stöd för bland annat symbolisk exekvering
skulle i praktiken innebära implementering av en emulator. Det är en relativ
stor och tidskrävande uppgift som dessutom kräver nogrann testning för att vara
korrekt och tillförlitlig. För att fastställa korrekthet ska applikationen
därför utnyttja existerande verktyg för att exekvera programmet med stöd för
symbolisk exekvering. Detta möjliggör också bredare plattformssupport jämfört
med en hemmasnickrad emulator.


\subsection{Verktyg}

\stoe~\cite{s2e} är en plattform för symbolisk exekvering som bygger på QEMU:s
virtuella maskin och använder KLEE~\cite{klee}, en motor för symbolisk
exekvering, som interpreter för att möjliggöra symbolisk exekvering. \stoe\ är i
sin tur utbyggbart med möjlighet för användaren att skriva ett eget plugin för
att utföra analyser och används inom säkerhetsforskning för att till exempel
analysera skadlig kod. \stoe\ användes som del av Galactica-systemet som spelade
i DARPA Cyber Grand Challenge~\cite{s2e_website}. \stoe\ är öppen källkod,
väldokumenterat och underhålls aktivt.

Ett alternativt verktyg för symbolisk exekvering är symQEMU~\cite{symqemu},
som också kombinerar QEMU:s virtuella maskin med KLEE:s motor för symbolisk
exekvering. Till skillnad från \stoe\ kompilerar SymQEMU KLEE in i den
analyserade binären och har jämförelsevis hög prestanda. Däremot har SymQEMU
bristfällig dokumentation och är ej aktivt uppdaterat.

Då SymQEMU ej uppdateras aktivt och har bristfällig dokumentation kommer \stoe\
användas. Projektet avgränsas i och med att existerande verktyg (\stoe) kommer
användas istället för att bygga en motor för symbolisk exekvering från grunden.

\subsection{Begränsningar}

Att använda \stoe\ innebär att arbetet avgränsas till att, i praktiken, skapa
ett plugin som avlyssnar och styr motorn. Varken emulator eller motor ska
byggas och de uppgifter som ingår i att skapa en exekveringsmotor exkluderas.

Det innebär att fokus flyttas ifrån motorns tekniska detaljer till resterande
jobbet med att utveckla en användbar slutprodukt som bygger ut \stoe:s redan
existerande funktionalitet med ett grafiskt användargränsnitt och möjlighet att
stega igenom, analysera och interaktivt besluta om värden under exekvering.

Beslutet innebär också att applikationens utformning blir bunden till \stoe:s
tekniska begränsningar.
