%Detta projekt ämnar utveckla ett binäranalysverktyg för generell reverse
%engineering, det vill säga en applikation vars uppgift är att analysera binära
%program utan kännedom om källkoden utifrån ett datorsäkerhetsperspektiv.

Projektet ämnar utveckla ett verktyg för interaktiv visualisering av symbolisk
fuzzing som möjliggör utforskande binäranalys med hjälp av symbolisk
exekvering. Visualiseringen ska presenteras i ett grafiskt användargränssnitt
där användaren kan interaktivt stega igenom det binära programmet och
prioritera exekveringsstigar. Verktyget ämnar att, genom mänskligt fördelaktiga
representationer av programmets beteende, öka användarens abstrakta förståelse
av det.

Att kunna avgöra ett programs beteende utifrån endast exekverbar (binär) 
kod är viktigt när man undersöker skadeprogram eftersom dess källkod oftast är
okänd. För att upptäcka pågående, och motverka framtida attacker är det viktigt
att förstå hur skadeprogram är utformade och hur de beteer sig. 

Användbarheten av ett analysverktyg som möjliggör utforskande analys kommer
från kombinationen av datorns fördel i beräkningskraft och människans
intuition. Symbolisk fuzzing löser problemet av att inte behöva godtyckligt
gissa input som leder till att en specifik väg tas i ett program,
och nyttan från människans intuition nyttjas på samma sätt som med en
dekompilator -- genom analys som leder till ökad förståelse och insikt. Denna
kombinationen gör det möjligt att på ett användarvänligt sätt dra analytiska
slutsatser om en binär och öka ens abstrakta förståelse av det och därmed
potentiellt hitta sårbarheter.
