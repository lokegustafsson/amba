
\subsubsection{Mål}
Detta projekt ämnar utveckla ett binäranalysverktyg för generell reverse
engineering, det vill säga en applikation vars uppgift är att analysera binära
program utan kännedom om källkoden utifrån ett datorsäkerhetsperspektiv.
Verktyget ska ha en korrekt och testbar förståelse av det analyserade programmet
och ska samtidigt kunna kommunicera denna till användaren genom visualisering.
Visualiseringen ska presenteras i ett grafiskt användargränssnitt där användaren
kan interaktivt stega igenom det binära programmet och själv välja vilka beslut
som görs gällande exempelvis programhopp. Funktionaliteten i verktyget uppnås
med hjälp av symbolisk exekvering, och därigenom kombineras fördelarna i
automatiska och manuella analysmetoder.

% Utveckla syfte (förstå nyttan och det akademiskt relevanta i arbetet)
Verktyget ämnar att genom mänskligt läsbara representationer av programmets 
beteende, öka användarens abstrakta förståelse av det. Verktyget ska vara 
användbart för att öka användarens abstrakta förståelse av ett program då källkod 
saknas. Det vill säga verktyget ska hjälpa användaren att resonera om programmet 
generellt. 

\subsubsection{Användningsområden}
Att kunna avgöra ett programs beteende utifrån endast exkeverbar (binär) 
kod är viktigt när man undersöker potentiellt skadlig mjukvara eftersom
dess källkod oftast är okänd. För att upptäcka pågående, och motverka 
framtida attacker är det viktigt att förstå hur attackerna är utformade och
hur de beteer sig. 

Binär analys är också användbart då tredje-parts bibliotek används.
Att analysera säkerheten hos ett program kan behöva ske utan att involvera 
dess utgivare. Då är det användbart att kunna dra slutsatser om programmet 
endast ifrån dess maskinkod. 

Förutom att analysera program där källkoden inte är tillgänglig så är det även 
användbart att analysera ett programs maskinkod för att undersöka kompilatorbuggar.

%\section{akademisk relevans}




