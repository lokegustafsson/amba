%Detta projekt ämnar utveckla ett binäranalysverktyg för generell reverse
%engineering, det vill säga en applikation vars uppgift är att analysera binära
%program utan kännedom om källkoden utifrån ett datorsäkerhetsperspektiv.

Projektet ämnar utveckla binäranalysverktyget AMBA för interaktiv visualisering
av symbolisk fuzzing som möjliggör utforskande binäranalys med hjälp av
symbolisk exekvering. Visualiseringen ska presenteras i ett grafiskt
användargränssnitt där användaren kan interaktivt följa exekveringen och
prioritera exekveringsstigar. Verktyget ämnar att, genom mänskligt fördelaktiga
representationer av programmets beteende, förbättra kommunikationen till
användaren och underlätta binäranalys inom skadeprogramsanalys och
sårbarhetsanalys.

Detta för att möjliggöra ett utforskande analysverktyg som kombinerar datorns
fördel i beräkningskraft och människans intuition. Symbolisk fuzzing löser
problemet av att inte behöva godtyckligt gissa input som leder till att en
specifik väg tas i ett program, och nyttan från människans intuition nyttjas på
samma sätt som med en dekompilator --- genom analys som leder till ökad
förståelse och insikt. Denna kombinationen gör det möjligt att på ett
användarvänligt sätt dra analytiska slutsatser om ett program i
maskinkod-format och öka ens abstrakta förståelse av det och därmed underlätta
binäranalys baserad på symbolisk fuzzing.
