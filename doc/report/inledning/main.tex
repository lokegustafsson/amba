Attacker mot organisationers eller nationers IT-system har ofta förrödande
direkta och indirekta konsekvenser för de drabbade. Enligt företaget
Cybereason\footnote{Cybereason är ett cybersäkerhetsteknikföretag som erbjuder
plattform för skydd av ändpunkter (jfr. eng. \emph{endpoints}) som t.ex.
antivirusprogram. Företaget utför även forskning inom cybersäkerhet genom
\emph{reverse engeneering} av skadeprogram och
systemsårbarhetsanalys~\cite{enwiki:1147596623}.} svarade över 70 procent av de
över 1400 tillfrågade säkerhetsspecialister i en
av deras studier att deras organisation hade utsats för
utpressningsprogram (jfr eng. \emph{ransomware}) minst en gång de senaste
24 månaderna~\cite{cyberreason2021, cyberreason2022}. Cybereasons rapporter
beskriver att några av de direkta följderna av att en lyckad attack är: höga
kostnader för organisationen; stoppad produktion; konsekvenser från reglerare;
och skadat anseende. Dessutom leder de höga kostnaderna till
bland annat uppsägningar av anställda för att kunna kompensera. Bloomberg
Newsweek~\cite{gallagher2023} beskriver hur cancerpatienter som behövde akut
sjukvård förflyttades till sjukhus med fungerade datorsystem när Irlands
offentliga sjukhussystem låg nere~\cite{hse_report}\todo{why is this needed?
This sentence talks about Newsweek's article only /Iulia}. Det finns olika
metoder till cyberattacker men många angripare och skadeprogram använder sig av
olika säkerhetsbrister i programvaror.

Vanliga exempel på säkerhetsbrister som utnyttjas av angripare och skadeprogram
är minnessårbarheter, som tillåter access till minne utanför gränserna till
viss minnesregion, t.ex. buffertöverflöden (jfr eng. \emph{buffer
overflow}) och formatsträng (jfr eng. \emph{format string}). Den
vanligaste minnessårbarheten är buffertöverflöde vilket uppstår när
programmerare inte utför tillräckliga gränskontroller, vilket
utlöser en åtkomst bortom gränserna för viss minnesregion. Angripare kan
utnyttja detta för att korrumpera programmets normala beteende genom att skriva
bortom gränserna~\cite{computer_security_cs161}. Formatsträngsbugg uppstår när
någon typ av indata tolkas och används som input till vanliga
textkonverteringsfunktioner som \texttt{printf}, som i sin tur använder
konvertingsfunktioner för data med olika typer. Om inte en noggrann och
tillräcklig utvärdering av indatan görs innan indatan används i en funktion som
kan tolka formatsträng kan funktionen använda stacken och den data som råkar
ligga där som argument för konverteringsfunktioner. Med noggrann utformning av
indata kan en angripare utnyttja formatsträngsbugg för att exekvera godtyckliga
kommandon på en dator som kör programvaran.

För att upptäcka sårbarheter i programvaror kan många 
