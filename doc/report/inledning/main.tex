utpressningsprogram (jfr. eng. \emph{ransomware attacks}) fördubblades
mellan 2021 och 2022 och 73\% av organisationer har varit utsatta för dem
minst en gång inom 24 månader~\cite{cyberreason2021,
    cyberreason2022}. Till exempel påverkades Irlands offentliga sjukhussystem
kraftigt när utpressningsprogram (jfr eng. \emph{ransomeware}) tog över en stor del av deras digitala
infrastruktur~\cite{hse_report, gallagher2023}. Några av de direkta påföljderna
av att en lyckad attack är: höga kostnader för organisationen; stoppad
produktion; juridiska konsekvenser; och skadat
anseende~\cite{cyberreason2021, cyberreason2022}. Dessutom leder de höga
kostnaderna till bland annat uppsägningar av anställda för att
kompensera~\cite{cyberreason2021, cyberreason2022}. I många fall är människan
den svaga länken i cyberattacker men skadeprogram (jfr eng.
\emph{malware}) utnyttjar olika sårbarheter i programvaror för uppnå delmål.
Exemplvis uttnyttjade utpressningsprogrammet WannaCry en
sårbarhet i ett populärt nätverksfildelningsprotokoll som tillät \emph{remote
    code execution} för att sprida sig mellan
datorer~\cite{alraddadicomprehensive}.

Genom att analysera program kan säkerhetsforskare och utvecklare upptäcka och
åtgärda sårbarheter innan en angripare. Programanalys går ut på att analysera
programegenskaper för att kontrollera korrekthet och hitta möjliga sårbarheter
och kan utföras olika på nivåer: på källkoden eller på maskinkoden (jfr eng.
\emph{binary}). Inom skadeprogramsanalys (jfr eng. \emph{malware analysis}) och
\emph{reverse engeneering} är binäranalys nödvändig eftersom programmet som ska
analyseras oftast endast tillgänglig som maskinkod~\cite{andriesse2018}.

% Programanalys innebär att man analyserar ett program för att säkerställa att
% vissa egenskaper uppfylls. Analysen kan göras på olika nivåer (källkoden eller
% maskinkoden (jfr eng. \emph{binary})) och har flera kategorier: statisk eller
% dynamisk och manuell eller automatisk.

Däremot är binäranalys (jfr eng. \emph{binaryanalysis}) ofta komplicerat och
tidskrävande. Skadeprogram tilltar ofta flera metoder för att försvåra
analys och vilseleda analysverktyg. Därför krävs en utforskande metod för att
analysera och förstå ett skadeprogram och testa ett antal metoder för att förstå
och känneteckna sårbarheter som skadeprogrammet utnyttjar. Detta kräver ett
interaktivt och utforskande analysverktyg som tillåter användaren att
känneteckna problematik i en analys och styra analysen åt ett håll som medför
resultat.

Symbolisk exekvering, beskrivet i detalj i avsnitt~\ref{sec:symbolic_execution},
är en kraftfull teknik som har möjliggjort stora mängder
binäranalysmetoder\cite{survey_symb_exc, symnav}. Inom skadeprogramsanalys
tillåter det att ``upptäcka sårbarheter, kringgå autentisering eller
skyddssystem, eller förstå deras inre delar''~\cite{symnav}. Det är en ett sätt
att exekvera ett program abstrakt där en exekvering betraktar alla möjliga
indata som delar en viss exekveringsstig (jfr eng. \emph{execution path}) genom
programflödet. Detta möjliggörs genom att betrakta indatan som symbolisk där en
viss stig ackumulerar stigvillkor (jfr eng. \emph{path conditions}) på symbolen
vars lösning/lösningar beskriver alla värden på symbolen som leder till att
programmet följer motsvarande stig i programflödet. Trots kraftfullheten i
symbolisk exekvering, finns det ett antal problem med tekniken. Det är en metod
som oftast leder till en långsam exekvering och begränsas även av
stigexplosionsproblemet (jfr eng. \emph{path explosion}), som orsakas av ett
stort antal villkorstest på symbolisk data, i bl.a.\ loopar, rekursion och
undantag (jfr eng. \emph{exceptions}). Det finns ett antal metoder för att
kringgå stigexplosionsproblemet men lägger stor börda på analytikern som ofta
har ett väldigt begränsad vy in i vad som händer under exekveringen, oftast
genom en textuell gränssnitt. Dessa problem kan utnyttjas av skadeprogram för
att vilseleda binäranalysmetoder som förlitar sig på symbolisk exekvering.

I den här rapporten föreslår vi AMBA för att, genom bättre kommunikation till
användaren, skapa möjligheter för användaren att känneteckna stigexplosion och
tillåta styrning bort från dessa stigar som orsakar problem. AMBA kommer även
att undersöka alla stigar i programflödet vilket är känd som symbolisk fuzzing.

SymNav~\cite{symnav}, som beskrvis i mer detalj i
avsnitt~\ref{sec:existerande-visualiserad}, har tidigare undersökt samma problem och
föreslagit liknande lösningar men har begräsningar i form av prestanda.

% En metod för binäranalys är \emph{fuzzing} som går ut på att mata olika indata
% till ett program och betrakta fel som uppstår. Målet med \emph{fuzzing} är att
% uppnå en hög täckning (jfr eng. \emph{coverage}) av programmet som beror
% på indatan. Alltså att ta alla möjliga vägar i programflödet som är möjliga att
% nå genom att variera indatan. Däremot är det ett svårt problem att generera
% indata som uppnår en hög täckning som även är effektivast, speciellt när
% indatan kan ha ett högt antal olika värden.

% * Svaga argument för någon som vet binäranalys: ransomware ofta
% människoproblem. Kommer inte till kärnan tills mycket senare.
% * Give example where ransomware uses såbarheter
% * legacy code main reason, not used to rust for example, too small memory to
% run high level programs?

% * Use this form in the introduction: Situation - Problem - Solution -
% Evaluation
% * use case: governments need to evaluate safety of products custom made for
% them
%
% * evaluation: result and process discussion - limitation - future work
% * conclution: results and contribution discussion. conclude results discuss why good for the society.
