Programförståelse är processen där människan får kunskap om hur ett datorprogram
eller mjukvarusystem fungerar~\cite{xia2018}. Programförståelse är ofta, helt
eller delvis, en informell process. Programförståelse används till exempel när
nyanställda läser dokumentation för att förstå hur ett mjukvarusystem är
uppbyggt. Ett annat exempel är när säkerhetsforskare undersöker skadlig
programvara (jfr.\ eng. \emph{malware}) för att kunna utveckla
motmedel~\cite{andriesse2018}. I fortsättningen använder vi ordet förståelse som
en förkortning av programförståelse.

För att bilda sig en förståelse av ett program formulerar man ofta hypoteser om
programmet, för att sedan testa om hypotesen stämmer~\cite{hermans2021}. Dessa
hypoteser kan vara övergripande, som till exempel om programmet kan utgöra ett
säkerhetshot.  Hypoteserna kan även vara specifika och fokuserade på detaljer,
som till exempel om en specifik formell specifikation är implementerad på ett
korrekt sätt eller vilka filsystemkataloger programmet använder sig av.

\section{Metoder för datorn}

Vissa hypoteser lämpar sig för att undersökas av en dator som effektivt kan
utföra automatiserade tester. Till exempel kan man ha implementerat en
sorteringsalgoritm som ska sortera olika värden i en specifik ordning. En
hypotes är då att implementationen av algoritmen är korrekt. Hypotesen kan då
testas genom att skapa ett antal testfall som kontrollerar att utdatan blir
korrekt. Däremot kan testning i det generella fallet aldrig verifiera att ett
program är helt korrekt.

En testmetod som automatiskt kan generera testfall och potentiellt hitta buggar
är fuzzing~\cite{macel2021}. Fuzzing går ut på att generera indata för att sedan
undersöka om indatan kan orsaka icke önskvärt beteende eller krascher hos
programmet~\cite{macel2021}. Ett sätt att generera indata inom fuzzing är att
betrakta typen på indatan och försöka testa olika kategorier som kan orsaka
felaktiga beteenden. Om indatan exempelvis är en numerisk typ kan man generera
ett litet värde, ett stort värde och ett negativt värde för att testa. Men om
endast ett specifikt värde leder till ett visst beteende är det sannolikt att
detta värde aldrig gissas och därmed inte inträffar.

Symbolisk fuzzing är en variation på fuzzing som utforskar flera möjliga vägar
genom ett program utan att köra det med alla konkreta värden som
indata~\cite{fuzzingbook2023:SymbolicFuzzer}.  Istället för att använda konkreta
värden används variabler för att symbolisera indata och programtillstånd. Genom
att spåra villkor för dessa variabler kan man generera testfall som täcker olika
vägar genom programmet. Målet är att undersöka så många vägar som möjligt för
att utforska programmets olika möjliga tillstånd som kan nås med olika indata.

För att demonstrera detta kan vi betrakta funktionen i figur
\ref{fig:symb_exc_ex_inl}. I funktionen kan det uppkomma en krasch om indatan
$x$ har värdet 2 eftersom nämnaren i returuttrycket $\frac{1}{2 - x}$ blir 0 och
således leder till en division med 0. Med symbolisk fuzzing representeras $x$
som en symbol istället för ett konkret värde. Därefter exekveras funktionen rad
för rad. Vid villkoret $2 \le x < 10$ övervägs båda möjligheter: när villkoret
är sant och när det är falskt. Symboliskt sett betyder det att vi har två
exekveringsvägar: vägen där villkoret är sant, där $2 \le x<10$, samt vägen där
villkoret är falskt, där $x < 2$ eller $x \ge 10$. Den andra vägen avslutas när
funktionen returnerar $x$. Men den första vägen delas ytterligare när
divisionen ger ett division-med-noll-undantag för $x=2$ och returnerar den
beräknade kvoten för övriga $x$.

\begin{figure}[h!]
    \centering
    \begin{lstlisting}[language=Python]
def foo(x):
    if  2 <= x < 10:
        return 1 / (2 - x)
    return x
	\end{lstlisting}
    \caption{
        Exempelfunktion som beräknar $\frac{1}{2 - x}$ om $2 \le x < 10$, annars returnerar $x$.
        Ett division-med-noll-undantag uppstår om $x = 2$}
    \label{fig:symb_exc_ex_inl}
\end{figure}

För program med tillräckligt mycket indata sker en så kallad kombinatorisk
explosion där en fullständig sökning av möjlig indata ej kan genomföras då
antalet fall växer exponentiellt med storleken på indatan. Ett motsvarande
problem, \textit{vägexplosionsproblemet}, uppstår i symbolisk fuzzing eftersom
antalet möjliga vägar genom programmet kan växa exponentiellt med längden på den
längsta vägen.

\section{Metoder för människan}

Andra typer av hypoteser kan vara lämpliga att undersökas av en människa, då en
människa kan använda sina erfarenheter och kunskaper för att se mönster, dra
slutsatser samt bilda sig en övergripande förståelse av ett program. Till
exempel kan en människa dra slutsatser från testresultat från manuella eller
automatiska tester. Om en bugg hittas vid testning kan en människa med
kunskaper inom IT-säkerhet utvärdera hur säkerhetskritisk buggen är. Ett annat
exempel på när det är lämpligt att en människa undersöker en frågeställning är
när metoden eller processen som används är svår att automatisera. En process
som är svår att automatisera är \emph{reverse engineering} som handlar om att
rekonstruera en modell av programmets arkitektur genom deduktiva resonemang för
att kunna exempelvis återskapa källkoden för ett program från
binärkoden~\cite{on_rev_eng}. \emph{Reverse engineering} används exempelvis
inom skadeprogramanalys då man ofta inte har tillgång till skadeprogrammets
källkod.

\section{Problem och motivation}

I de flesta existerande verktyg för att analysera program är det antingen
människan eller datorn som testar hypoteser. Existerande verktyg där
framförallt människan testar hypoteser präglas av mycket manuellt arbete. Detta
innefattar verktyg från dekompilatorer där människan läser pseudokod konstruerad
utifrån maskinkoden till debuggers där människan kör programmet steg för steg.
Existerande verktyg där framförallt datorn testar hypoteser präglas av att
övergripande insikter missas, av kombinatoriska explosioner såsom
vägexplosionsproblemet samt att potentiell programförståelse går förlorad i
klumpig kommunikation med användaren. Detta innefattar bland annat de fuzzers
som endast ger användaren indata som orsakar vissa specifika beteenden.

Människan och datorn har olika styrkor när det kommer till att undersöka de
hypoteser man tagit fram för att öka förståelsen. I sammanhang där det är
kritiskt att bilda en djup förståelse är det därför intressant att ge människan
möjlighet att samarbeta med datorn för att kunna utnyttja bådas fördelar.
Samarbete mellan människan och datorn kan leda till en mer komplett förståelse
som i sin tur kan leda till åtgärdade säkerhetshål och buggar.

För att till en människa effektivt förmedla den data som en dator genererar i
sin analys behövs ett verktyg som översätter kopiösa mängder data till ett
mänskligt överskådligt format. Genom att visualisera denna data kan datorn
utöver att kommunicera detaljer även bidra till en mer generell
programförståelse på högre abstraktionsnivå. Om verktyget dessutom är
interaktivt kan människan med sin generella programförståelse styra datorns
analys på ett sätt som undviker fallgropar och ytterligare gynnar den abstrakta
förståelsen.

Vi tror att interaktiv visualiserad symbolisk fuzzing, genom att utnyttja både
datorn och människans styrkor, kan komplettera existerande verktyg.

\section{Mål}\label{sec:mal}

Projektet ämnar utveckla AMBA (\emph{Automatisk och Manuell BinärAnalys}), ett
verktyg som använder symbolisk fuzzing för att analysera ett program på
maskinkodsnivå. Den information AMBA utvinner ska presenteras visuellt för
användaren i ett grafiskt gränssnitt. Genom det grafiska gränssnittet ska
användaren kunna påverka de beslut AMBA tar i sin vägutforskning.

Målet med AMBA är att genom interaktiv visualisering låta användaren övervaka
symbolisk fuzzing.

\section{Avgränsningar}

AMBA kommer att fokusera på binäranalys och inte analys av program på
källkodsnivå.

AMBA ska demonstrera potentialen i interaktiv visualiserad symbolisk fuzzing,
men behöver inte nödvändigtvis uppnå denna potential genom att vara ett
praktiskt verktyg.

AMBA har inte det primära syftet att angripa vägexplosionsproblemet utan
fokuserar på användbar visualisering.

\section{Rapportstruktur}

Avsnitt~\ref{chap:teori} beskriver bakgrund och förklarar teori i mer detalj.
Efterföljande avsnitt (avsnitt~\ref{chap:existerande_verktyg}) beskriver
tidigare arbete och kategorisering av olika existerande binäranalysverktyg. I
avsnitt~\ref{chap:amba} beskrivs binäranalysverktyget AMBA och dess
funktionalitet och efterföljs med resultat och implementationsbeskrivning i
avsnitt~\ref{chap:implementation}. Avsnitt~\ref{chap:evaluering} tar upp och
evaluerar projektets resultat, process, begräsningar och framtida arbete och
förbättringar som kan göras. Evalueringen sker i form av en kvalitativ
jämförelse av funktionalitet i AMBA och existerande verktyg. Slutligen
sammanfattas och diskuteras projektets bidrag i avsnitt~\ref{chap:slutsats}.
