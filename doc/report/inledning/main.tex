Antalet cyberattacker med utpressningsprogram (jfr. en. \emph{ransomware}) fördubblades mellan 2021 och 2022 och 73 procent av organisationer har varit utsatta för dem minst en gång inom 24 månader~\cite{cyberreason2021, cyberreason2022}.
Till exempel påverkades Irlands offentliga sjukhussystem kraftigt när utpressningsprogram tog över en stor del av deras digitala infrastruktur~\cite{hse_report, gallagher2023}.
Några av de direkta påföljderna av en lyckad attack är: höga kostnader för organisationen; stoppad produktion; juridiska konsekvenser; och skadat anseende~\cite{cyberreason2021, cyberreason2022}.
Dessutom leder de höga kostnaderna till bland annat uppsägningar av anställda för att kompensera~\cite{cyberreason2021, cyberreason2022}.
Utpressningsprogram är program som hindrar legitima användare att komma åt datorsystem genom att låsa dem~\cite{accessscience_computer_virus}.
Cyberattackerarna kräver sedan en lösensumma för att låsa upp systemen.
I många fall är människan den svaga länken i cyberattacker med utpressningsprogram.
Ibland krävs det ingen mänsklig interaktion för att utpressningsprogram ska sprida sig inom och mellan datorsystem.
Till exempel uttnyttjade utpressningsprogrammet WannaCry en sårbarhet i ett populärt fildelningssystem på Windowsdatorer~\cite{alraddadicomprehensive}.
Sårbarhenten möjliggjorde för WannaCry att genom en fjärrkodsexekveringsattack sprida sig automatiskt mellan datorsystem.
Fjärrkodsexekvering är en klass av cyberattacker där attackerare kan exekvera sin egna kod på måldatorn utan någon mänsklig intaraktion~\cite{baker2022}.

Programanalys går ut på att analysera egenskaperna hos ett program för att kunna kontrollera programets korrekthet och hitta möjliga sårbarheter och kan utföras olika på nivåer: på källkoden eller på maskinkoden.
Genom att analysera program kan säkerhetsforskare och utvecklare upptäcka och åtgärda sårbarheter innan de hinner utnyttjas av en angripare.
Inom skadeprogramsanalys (jfr eng. \emph{malware analysis}) och \emph{reverse engeneering} är binäranalys nödvändig eftersom programmet som ska analyseras oftast endast tillgänglig som maskinkod~\cite{andriesse2018}.

Binäranalys av skadliga programvaror är ofta komplicerat och tidskrävande då utvecklarna av skadlig programvara ofta ofta använder olika obfuskeringsmetoder för att försvåra analyseringen av den skadliga programvaran.
Därför krävs en utforskande metod för att analysera och förstå ett skadeprogram och testa ett antal metoder för att förstå och känneteckna sårbarheter som skadeprogrammet utnyttjar.
Detta kräver ett interaktivt och utforskande analysverktyg som tillåter användaren att känneteckna problematik i en analys och styra analysen åt ett håll som medför resultat.

Symbolisk exekvering är en teknik som har möjliggjort stora mängder binäranalysmetoder\cite{survey_symb_exc, symnav}.
Inom skadeprogramsanalys tillåter det att ``upptäcka sårbarheter, kringgå autentisering eller skyddssystem, eller förstå deras inre delar''~\cite{symnav}.
Det är en ett sätt att exekvera ett program abstrakt där en exekvering betraktar alla möjliga indata som delar en viss exekveringsstig (jfr eng. \emph{execution path}) genom programflödet.
Detta möjliggörs genom att betrakta indatan som symbolisk där en viss stig ackumulerar stigvillkor (jfr eng. \emph{path conditions}) på symbolen vars lösning/lösningar beskriver alla värden på symbolen som leder till att programmet följer motsvarande stig i programflödet.
Trots kraftfullheten i symbolisk exekvering, finns det ett antal problem med tekniken.
Det är en metod som oftast leder till en långsam exekvering och begränsas även av stigexplosionsproblemet (jfr eng. \emph{path explosion}), som orsakas av ett stort antal villkorstest på symbolisk data, i bl.a.\ loopar, rekursion och undantag (jfr eng. \emph{exceptions}).
Det finns ett antal metoder för att kringgå stigexplosionsproblemet men lägger stor börda på analytikern som ofta har ett väldigt begränsad vy in i vad som händer under exekveringen, oftast genom en textuell gränssnitt.
Dessa problem kan utnyttjas av skadeprogram för att vilseleda binäranalysmetoder som förlitar sig på symbolisk exekvering.

I den här rapporten föreslår vi AMBA för att, genom bättre kommunikation till användaren, skapa möjligheter för användaren att känneteckna stigexplosion och tillåta styrning bort från dessa stigar som orsakar problem. AMBA kommer även att undersöka alla stigar i programflödet vilket är känd som symbolisk fuzzing.

SymNav~\cite{symnav}, som beskrvis i mer detalj i avsnitt~\ref{sec:existerande-visualiserad}, har tidigare undersökt samma problem och föreslagit liknande lösningar men har begräsningar i form av prestanda.