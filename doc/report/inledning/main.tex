Attacker mot organisationers eller nationers IT-system har ofta förrödande
direkta och indirekta konsekvenser för de drabbade. Enligt företaget Cybereason
svarade över 70 procent av de över 1400 tillfrågade säkerhetsspecialister i en
av deras studier att deras organisation hade utsats för
\emph{Utpressningsprogram} (engelska: \emph{ransomware}) minst en gång de senaste
24 månaderna~\cite{cyberreason2021, cyberreason2022}. Cybereasons rapporter
beskriver att några av de direkta följderna av att en lyckad attack är höga
kostnader för organisationen; stoppad produktion; konsekvenser från reglerare;
och skadat anseende. Enligt Cybereasons rapporter leder de höga kostnaderna till
bland annat uppsägningar av anställda för att kunna kompensera. I en artikel i
Bloomberg Newsweek~\cite{gallagher2023} beskrivs hur cancerpatienter som behövde
akut sjukvård förflyttades till sjukhus med fungerade datorsystem när Irlands
offentliga sjukhussystem låg nere~\cite{hse_report}. Ransomware attacker
möjliggörs tack vare säkerhetsbrister i programvarorna som finns på offrens
datorsystem.

