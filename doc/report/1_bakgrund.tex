% BAKGRUNDENS UPPGIFT?: Motivera behovet att visualiseringsverktyg för fuzzing
% (fuzzing används slarvigt, inkludera tex concolic testing)

Att läsa källkod är ett sätt att förstå program, men ibland är det gynnsamt att istället betrakta
maskinkoden direkt. Detta kan vara för att
\begin{itemize}
  \item utesluta påverkan av kompilatorbuggar som ger oväntad maskinkod
  \item se hur UB (\emph{undefined behaviour}) har utnyttjats av kompilatorn
  \item källkoden inte är tillgänglig
\end{itemize}

Det finns en uppsjö av metoder som kan användas för att analysera en exekverbar
binär. Exempel på dessa är att:
\begin{enumerate}
  \item disassemblera binären och läsa dess funktioner för att förstå vad de gör.
  \item dekompilera assemblykoden med ett verktyg som ger pseudokod, och sedan läsa denna mer
    läsbara koden.
  \item köra binären på speciella testfall och jämföra svaret med vad som förväntas. Om
    programmet implementerar en specifikation kan en existerande testsamling användas.
  \item fuzztesta binären, det vill säga automatiskt generera testfall tills ett orsakar en crash eller
    annat oönskat beteende i binären. Många fuzztestmotorer skapar testfall med en evolutionär
    algoritm, och många använder instrumentering över vilka programhopp som tas för att bedöma
    testfalls nyttighet.
  \item använda concolic testing, alltså fuzzing där en SMT solver genererar nya testfall genom att
    lösa för testfall som orsakar annorlunda programhopp.
  \item stega igenom programmet i en debugger för att se exakt vad programmet gör med viss input.
\end{enumerate}

För att bilda en allmän förståelse om programmet krävs både \textit{korrekt} och
\textit{abstrakt} förståelse. I detta avseende syftar \textit{korrekt} på
avsaknaden av felaktiga slutsatser och \textit{abstrakt} på möjligheten att
resonera om programmet generellt i motsats till att resonera om en specifik
konkret indata i taget.

Metod 1-2, att läsa kod, kan ge en \textit{abstrakt} förståelse av vad
programmet gör, men för att verifiera att huruvida resonemanget är korrekt krävs
hypotestestning vilket kräver att programmet körs. Således går det inte att
bilda en \textit{korrekt} förståelse genom att enbart läsa kod.

Metod 3-5, att köra programmet på testfall, ger framförallt en
black-box-förståelse av programmet. Tillgången till binären och
exekveringsmiljön används endast som ett verktyg för att generera nya testfall.
Fuzzing och concolic testing kan köras helautomatiskt och är \textit{korrekta}.
Men ofta är en tillräckligt täckande sökning av indatarummet omöjlig, och då kan
den automatiska analysen ha missat ett kvalitativt annorlunda beteende. Dessutom
ger en omfattande uppsättning indata-utdata-par inte användaren samma
information som källkoden ger. Därmed är helautomatiska analysmetoder inte
\textit{abstrakta}. Notera att det inte nödvändigtvis tyder på en brist i den
automatiska analysen att ett kvalitativt annorlunda beteende missas, för det
gömda beteendet skulle kunna vara en konsekvens av komplicerad kod, som till
exempel ett hoppvilkor beroende på en kryptografisk hash av indatan. Men en
analysmetod borde kunna peka ut var dess förståelse tar slut, snarare än att
utelämna detta fullständigt vilket är vad avsaknaden av testfall visar sig som.

Med metod 6, en debugger, kan användaren följa exekveringen för en viss indata
utan att riskera att missförstå hur datan transformeras. Om användaren har ett
oändligt tålamod kan de göra detta om och om igen för olika indata genererade
med till exempel fuzzing. Varje genomstegning ger information om koden som
behandlar indatan men också viss information om övrig kod -- till exempel kan
ett svårtaget hopp indikera en plats för användaren att rikta sin uppmärksamhet
mot. Detta ger en både \textit{korrekt} och \textit{abstrakt} förståelse, men
med en orimlig manuell arbetsbörda för användaren.

En helautomatisk \textit{korrekt} metod kan ge en \textit{abstrakt} förståelse
om processens förlopp visualiseras för användaren. Valet mellan manuell
arbetsbörda som ger djup förståelse och en testfallsgenerationsdriven process
som ger översiktlig förståelse kan genomföras av användaren om verktygen stödjer
hela spektrummet.
