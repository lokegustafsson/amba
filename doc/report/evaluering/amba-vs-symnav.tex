SymNav är ett mer moget verktyg än AMBA.\@ Den största skillnaden mellan verktygen
är att SymNav presenterar informationen från den symboliska fuzzingen på ett mer
användarvänligt vis och att SymNav låter användaren filtrera bort de
exekveringsvägar användaren inte vill fokusera på.

Att AMBA saknar tillståndsfiltrering gör verktyget opraktiskt för icke-triviala
program, då visualiseringen blir överväldigande för användaren och AMBAs
tillståndsfiltrering otymplig. AMBA kan därför inte konkurrera med SymNav i
praktisk användbarhet, utan endast i det akademiska intresset i dess
särskiljande funktionalitet samt potentiellt i den vidareutveckling som
möjliggörs av AMBAs arktitektur.

En stor skillnad mellan AMBA och SymNav är att medan AMBA uppdaterar det
grafiska gränssnittet i realtid, eller mer precist en gång per sekund, fungerar
SymNav så att användaren startar körningar som varar så länge som specifierat av
användaren. I ett typiskt arbetsflöde i SymNav undersöker användaren programmet
genom det grafiska gränssnittet, startar en körning på en minut och väntar tills
den är klar och återgår först därefter till att undersöka programmet genom det
nu uppdaterade grafiska gränssnittet. En realtidsarkitektur möjliggör mer
interaktivitet i hur användaren påverkar den symboliska fuzzingen.

En annan stor skillnad är vilken symbolisk exekveringsmotor som används. SymNav
använder angr och AMBA använder \stoe{}. Den största konsekvensen av denna
skillnad är att AMBA har potentiellt bättre prestanda i den symboliska
exekveringen, vilket diskuterades i det förra avsnittet. Denna fördel kan, om
AMBA utökas med tillståndsfiltrering, vara en faktor som gör AMBA mer passande
än SymNav i något sökningstungt användsområde.
