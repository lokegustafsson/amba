% Vad ska vi evaluera?
    % jämförelse med andra verktyg, hur står sig vårt verktyg mot andra verktyg?
        % vi använder idén om att användare får ge input till verktyget för att
        % analysera specifika vägar, eller särskilda constraints.
        % prestandamässigt, hur skiljer sig vårt verktyg från andra white-box
        % fuzzers?
% Hur evaluerar vi det?

% Varför har vi valt att evaluera på detta sättet (rimligt att ta upp?)


% Vad har vi för features?
% node colouring, disassembly debuginfo (for every node?), state prioritization
% (hur får vi med det?)

% Text under titel
För att evaluera verktyget används existerande verktyg som jämförelse i
kombination med att dess fördelar och nackdelar vägs mot AMBA. Dessutom jämförs
funktionalitet -- AMBA jämförs med andra verktyg där likheter
och skillnader lyfts.

Ghidra är ett avancerat verktyg som gör statisk analys bortom syftet med AMBA,
men AMBA har ett par likheter. Ghidra kan representera en
kontrollflödesgraf för en given binär på två olika sätt: \textit{Flow
Graph} och \textit{Code Flow Graph}. Ghidra kompletterar dessa vyer
med en \textit{code listing}, en primär vy där binärens
disassemblerade kod listas. Användaren kan fritt välja att bilda en
graf från ett markerat \textit{code block}, vilket är Ghidras utökade
specifikation på ett basic block, från Ghidras code listing. 




% ^vi gör också detta, fast på ett annat sätt och med fler alternativ (eller
% sätt att visualisera grafen med olika typer av grafer).

% Vad gör vi bättre?
% Vad gör vi sämre?
% Vad gör vi annorlunda?
% Hur ska denna del struktureras?

%\subsection{AMBA vs angr}
%\subsection{AMBA vs Ghidra} 
%\subsection{AMBA vs SymQEMU}
%\subsection{AMBA vs SAGE}
%\subsection{Vidareutveckling}
% Lägga till future work, vad behandlar inte den här rapporten, vad hade vi
% velat att verkyget kunde göra? osv

