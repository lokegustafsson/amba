% result and process discussion - limitation - future work  
För att evaluera verktyget används existerande verktyg som jämförelse i
kombination med att dess fördelar och nackdelar vägs mot AMBA. Dessutom
presenteras en diskussion kring utvecklingsprocessens gång och
vidareutvecklingspotential. 

\section{Jämförelse mellan AMBA och Ghidra} Ghidra är ett avancerat verktyg som
gör statisk analys bortom syftet med AMBA, men AMBA har ett par likheter. Ghidra
kan representera en kontrollflödesgraf för en given binär på två olika sätt:
\textit{Flow Graph} och \textit{Code Flow Graph}~\cite{ghidra_website} **FIGUR
HÄR**. Ghidra kompletterar dessa vyer med en \textit{code listing}, en primär vy
där binärens disassemblerade kod listas **figur här (typ en a,b,c figur)**.
Användaren kan fritt välja att bilda en graf från ett markerat \textit{code
block}, vilket är Ghidras utökade specifikation på ett basic block, från Ghidras
code listing~\cite{ghidra_website}. AMBA har liknande funktionalitet och kan
visualisera binärens fullständiga disassemblerade kod för en given nod **REF
TILL FIGUR HÄR (lokes)** men saknar större kontext likt Ghidras code listing. En
fördel mot Ghidras representation är AMBAs tre olika sätt att representera en
binärs kontrollflöde på: \textit{Raw Basic Block Graph}, \textit{Compressed
Block Graph}, och \textit{State Graph}. Genom att använda dessa tre olika vyer
kan en användare bilda en större förståelse om en given binär, dels genom att
exempelvis visualisera alla symboliska tillstånd som S2E skapar i kombination
med färgade noder som indikerar på starkt kopplade komponenter (jfr. eng.
\emph{strongly connected components}).

Kännedom om starkt kopplade komponenter i en kontrollflödesgraf är i synnerhet
intressant vid analys av binärer utan källkod, eftersom dessa delgrafer speglar
en starkare koppling mellan noderna och indikerar på en sektion i binären som
troligtvis har större relevans än en sektion som saknar eller har betydligt
färre starkt kopplade komponenter. 

En användare skulle exempelvis kunna använda AMBAs node colouring för att bilda
en större uppfattning om en viss del i binären och sedan komplettera med
nodernas minnesaddresser för att visualisera samma sektion i ett mer avancerat
verktyg som Ghidra. Ett typiskt användningsfall skulle kunna vara att i Ghidra
söka efter strängar som ger mer information i den givna kodsektionen från AMBA
eller använda Ghidras analysmetoder för att dekompilera binärens instruktioner
till ett mer läsbart format i form av pseudokod. 

% Vad ska vi evaluera?
    % jämförelse med andra verktyg, hur står sig vårt verktyg mot andra verktyg?
        % vi använder idén om att användare får ge input till verktyget för att
        % analysera specifika vägar, eller särskilda constraints.
        % prestandamässigt, hur skiljer sig vårt verktyg från andra white-box
        % fuzzers?
% Hur evaluerar vi det?

% Varför har vi valt att evaluera på detta sättet (rimligt att ta upp?)


% Vad har vi för features?
% node colouring, disassembly debuginfo (for every node?), state prioritization
% (hur får vi med det?)

% Text under titel

% Ghidra does this
% Amba doesn't
% What does AMBA do?
% How is it different, what does this entail?

% walkthrough på ett exempel

% ^vi gör också detta, fast på ett annat sätt och med fler alternativ (eller
% sätt att visualisera grafen med olika typer av grafer).

% Vad gör vi bättre?
% Vad gör vi sämre?
% Vad gör vi annorlunda?
% Hur ska denna del struktureras?

%\subsection{AMBA vs angr}
%\subsection{AMBA vs Ghidra} 
%\subsection{AMBA vs SymQEMU}
%\subsection{AMBA vs SAGE}
%\subsection{Vidareutveckling}
% Lägga till future work, vad behandlar inte den här rapporten, vad hade vi
% velat att verkyget kunde göra? osv

