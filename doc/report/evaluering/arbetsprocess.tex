Eftersom \stoe{} är ett komplext system med många komplicerade
biblioteksberoenden beslutades det tidigt i arbetsprocessen att göra AMBA mer
tillgängligt till slutanvändaren genom att paketera alla bibliotek som krävs för
att köra \stoe{}. Detta var en komplicerad och tidskrävande uppgift eftersom
mycket av \stoe{}s byggsystem var tvunget att återimplementeras eller kringgås
när tvetydiga och icke-triviala problem uppstod.

Ett sådant problem var byggningen av diskavbilder till de virtuella maskiner som
\stoe{} kör inuti QEMU. Detta problem löstes aldrig fullständigt utan fick
arbetas runt genom att ladda ned färdiga diskavbilder byggda och uppladdade av
\stoe{}-projektet i Google Drive.

Ett annat problem, detta orsakat av utdaterade bygginstruktioner i \stoe{}, var
att QEMU-projektets repon pekades på genom utdaterade, nu trasiga länkar under
\texttt{git.qemu-project.org} istället för uppdaterade fungerande länkar under
\\\texttt{gitlab.com/qemu-project}.

Ett tredje problem var att biblioteket \texttt{libs2e} byggdes med ett implicit
beroende på biblioteket \texttt{libgomp} som distribueras tillsammans med
kompilatorn \texttt{gcc}. Specifikt hade komponenten \texttt{libs2e}-komponenten
KLEE ett beroende på vissa symboler i statisk länkning som inte var tillgängliga
från \texttt{libgomp} från Nix-paketerade \texttt{gcc}. Detta problem fick
arbetas runt genom att dynamiskt länka en Ubuntu-byggd \texttt{libgomp} till
\texttt{libs2e}.

Dessa problem, tillsammans med väldigt många fler av liknande natur, ledde till
mycket arbete som försköt planeringen. Således fanns det mindre tid till att
utveckla all funktionalitet som var tänkt i AMBA.
