En särskilt eftertraktad funktionalitet är tillståndssammanslagning. Syftet med
tillståndssammanslagning är att minska antalet exekveringsvägar genom att förena
symboliska tillstånd som är nästan ekvivalenta. Detta kan till exempel
implementeras genom att ge uppgiften till användaren att interaktivt välja vilka
tillstånd som ska sammanslås. Således ökar prestandan för den symboliska
exekveringen som gör det möjligt att skicka fler förfrågningar (jfr.\ eng.\
\emph{queries}) till SMT-lösaren.

Övervakning av systemanrop ger användaren en
större insikt i vad som sker i en specifik del av grafen, exempelvis om det sker
inmatning eller utmatning, om en process signaleras att stängas av, om binären
kör ett \texttt{exec}-systemanrop för att köra ett externt program, etc. Att det är
intressant att känna till programmets agerande i dess miljö gör det intressant
att övervaka systemanrop. Detta skulle kunna implementeras i AMBA genom att
lyssna efter exekvering av \texttt{sysmtemanrop}-instruktioner och när dessa
exekveras läsa de register som definerar vilket systemanrop som anropas och med
vilka parametrar det anropas.

Under AMBAs utvecklingsprocess har väldigt begränsad testning på icke-triviala
program genomförts. Detta innebär att AMBA är ytterst svåranvänt när programmets
kontrollflödesgraf är stor, både sett till att det grafiska gränssnittet saktar
ned men speciellt sett till att användaren inte har några verktyg att filtrera
bland informationen som presenteras. AMBA kan därmed vidareutvecklas genom att
implementera filter liknande SymNavs filtreringsstöd eller annan funktionalitet
lämpad för utforskning av icke-triviala program.
