En särskilt eftertraktad funktionalitet är tillståndssammanslagning. Syftet
med tillståndssammanslagning är att minska antalet exekveringsvägar genom att förena
symboliska tillstånd som är ekvivalenta. Ett förslag är att ge uppgiften till
användaren att interaktivt välja vilka tillstånd som sammanfogas. Således ökar
prestandan för den symboliska exekveringen som gör det möjligt att skicka fler
förfrågningar (jfr. eng. \emph{queries}) till SMT-lösaren.

Övervakning av systemanrop ger användaren en större
insikt i vad som sker i en specifik del av grafen, exempelvis om det sker
inmatning eller utmatning, om en process signaleras att stängas av, om binären
kör ett exec-systemanrop för att köra ett externt program, etc. är anledningar
som gör det intressant att övervaka systemanrop. För att implementera detta i
AMBA bör det undersökas om det finns hooks för detta.

I nuläget sparas inte resultatet för ett givet path constraint
och därför får användaren inte heller mer informatiom om vilken indata som ledde
till att vägen nåddes i exekveringen. Optimalt hade varit om användaren kunde se
vilken indata som ledde till en särskild väg för fortsatt analys. \stoe{} har
denna information, men i nuläget sparas den inte av AMBA. Dessutom hade det
varit intressant för användare om en SMT-lösare användes för att lösa ett givet
path constraint och därmed få indata som leder till att den vägen tas i
exekveringen.

