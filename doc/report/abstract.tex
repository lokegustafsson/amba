
This thesis project presents the development of AMBA, a binary analysis 
tool that utilizes symbolic execution and fuzzing to create an interactive 
visualisation of the program's execution. The objective of AMBA is to provide 
a way to analyze binary code and identify potential memory-related security 
vulnerabilities by combining the benefits of automatic computing power and 
human intuition. AMBA is created as a set of subsystems using \stoe{} to 
execute a binary symbolically. The methodology of the project involved 
developing a custom \stoe{} plugin to integrate symbolic execution with 
dynamic binary analysis. AMBA uses this plugin to collect information during 
symbolic execution in order to visualise the execution to the user as an 
interactive graph of the program's states. AMBA can be used by security 
researchers and developers to improve the security of their software 
applications.

% missing: 
% conclusion
% evaluation
% future work

keywords: symbolic execution, dynamic binary analysis, \stoe{}, 
fuzzing, vulnerability detection, memory vulnerabilities. 