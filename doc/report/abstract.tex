This thesis project introduces AMBA, an interactive tool for
visualising the execution of a binary program and identifying
potential memory-related security vulnerabilities by combining the
benefits of automatic computing power and human intuition. AMBA
combines symbolic execution and fuzzing to present the user with
graphs depicting execution paths in the program. These paths can be
explored interactively to identify potential vulnerabilities related
to memory usage. AMBA is better than similar existing tools in that it
visualises analysis in realtime, enables the user to prioritise states
and handles self-modifying code. AMBA could in the future be extended
with state-merging to enable analysis of larger programs with many
possible paths, as well as monitoring of system calls executed by the
binary and adding output of meaningful information resulting from the
symbolic fuzzing. AMBA is evaluated through a theoretical comparison
with existing tools with the conclusion that AMBA provides a useful
basis for binary analysis and has potential to become increasingly
beneficial if the suggested future work is implemented. Due to its
implementation as a plugin to the symbolic execution engine \stoe{}, AMBA
is readily available to developers and security researchers for
investigating the behaviour of machine code.

% something about demos?

Keywords: symbolic execution, binary analysis, \stoe{},
fuzzing, vulnerability detection, memory-related vulnerabilities.
