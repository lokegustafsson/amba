
This thesis project presents the development of AMBA, a binary analysis
tool that utilizes symbolic execution and fuzzing to create an interactive
visualisation of the program's execution. The objective of AMBA is to provide
a way to analyze binary code in order to identify potential memory-related security
vulnerabilities by combining the benefits of automatic computing power and
human intuition. AMBA is created as a set of subsystems using \stoe{} to
execute a binary symbolically. The methodology of the project involved
developing a custom \stoe{} plugin to integrate symbolic execution with
dynamic binary analysis. AMBA uses this plugin to collect information during
symbolic execution in order to visualise the execution to the user as an
interactive graph of the program's states. The graph allows the user to explore 
different paths through the program and identify potential vulnerabilities 
related to memory usage, such as buffer overflows. AMBA could in the future  
be extended with state-merging to enable analysis of larger programs with 
many possible paths, as well as monitoring of system calls executed by the binary and 
output of meaningful information resulting from the symbolic fuzzing. By 
providing an interactive visualization of the program's execution, AMBA can 
be used by security researchers and developers to identify vulnerabilities in 
software applications.

% missing: 
% conclusion
% evaluation

keywords: AMBA, symbolic execution, binary analysis, \stoe{},
fuzzing, vulnerability detection, memory-related vulnerabilities.