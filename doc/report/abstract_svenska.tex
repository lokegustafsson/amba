Det här kandidatarbetet introducerar AMBA, ett interaktivt verktyg för
visualisering av exekvering av binärprogram och identifiering av
potentiella minnesrelaterade säkerhetsbrister genom att kombinera
fördelarna av automatisk datorberäkning och männsklig intuition.  AMBA
kombinerar symbolisk exekvering och fuzzing för att presentera grafer
av exekveringsvägar i programmet för användaren. Dessa vägar kan
utforskas interaktivt för att identifiera potentiella säkerhetsbrister
relaterade till minnesanvändning. AMBA är bättre än liknande
existerande verktyg i och med att det visualiserar analysen i realtid,
vilket öppnar upp möjligheter för användaren att prioritera tillstånd
och hantera självmodifierande kod. AMBA kan i framtiden utvecklas med
tillståndssammanslagning för att möjliggöra analys av större program
med många möjliga vägar, och att övervaka systemanrop exekverade av
programmet och även lägga till utmatning av betydelseful information
som kommer från den symboliska fuzzingen. AMBA evalueras genom en
teoretisk jämförelse med existerande verktyg med slutsatsen att AMBA
ger en användbar grund för binäranalys och har potentialen att bli mer
användbar om de föreslagna vidareutvecklingarna implementeras.  Tack
vare implemenationens som ett plugin till den symboliska
exekveringsmotorn \stoe{} är AMBA lättillgängligt till utvecklare och
säkerhetsforskare som granskar beteendet av maskinkod.
