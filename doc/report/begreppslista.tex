\begin{labeling}{begreppslista}

    \item [\textbf{Assembler}] Assembler (jfr.\ eng.\ \emph{assembly}) är ett
    lågnivåspråk som direkt kan översättas till maskinkod.

    \item[\textbf{Basic block}] \emph{Basic block} (jfr.\ sv.\ grundblock) är en
    sekvens av instruktioner som saknar hopp eller förgreningar bortsett från
    början och slutet av blocket. Är oftast expanderade till att bli så långa
    som möjligt

    \item [\textbf{Black-box analys}] Black-box analys (jfr.\ sv.\
    svartlådeanalys) är en analys av ett objekt som endast betraktar dess
    yttre utseende och beteende, till skillnad från white-box-analys som även
    betraktar vad som händer inuti.

    \item [\textbf{Binär}] En binär (jfr.\ eng.\ \emph{binary}) är en fil som
    innehåller ett programs maskinkod och data.

    \item [\textbf{Återanropsfunktion}] Återanropsfunktion (jfr.\ eng.\
    \emph{callback function}) är funktioner som läggs i en \emph{hook} (jfr.\ sv.\ krok) för
    att exekveras vid ett visst tillstånd.

    % \item [\textbf{CFG}] (\emph{Control Flow Graph},
    %       jfr.\ sv.\ kontrollflödesgraf)

    % \item [\textbf{Konkolisk testning}] Konkolisk testning (jfr.\ eng.\ \emph{Concolic
    %     Testing}) är en sorts fuzztestning där indata genereras genom att med en SMT-
    % lösare lösa ekvationerna som beskriver en körning som ursprungligen följer
    % körningen för en seed-indata, men vid ett angivet hopp avviker. Detta är en
    % effektiv metod för att konstruera indata som tar speciella svårtagna hopp.
    %
    \item [\textbf{CTF}] CTF (\emph{Capture the Flag}, jfr.\ sv.\ fånga
    flaggan) är i datorsäkerhetssammanhang är en utmaning eller övning i att
    hitta gömda ''flaggor'' i program med avsiktliga säkerhetshål.

    \item [\textbf{Dynamisk analys}] Dynamisk analys (jfr.\ eng.\ \emph{dynamic
        analysis}) är när en binär analyserar genom att exevera binären i en
    kontrollerad miljö för att i detalj registrera vad binären gör.

    \item [\textbf{ELF}] ELF (\emph{Executable and Linkable Format}, jfr.\ sv.\
    exekverbart och länkbart format). Filformatet för exekverbara filer
    på Linux och liknande system. Innehåller maskinkod och länkar till andra
    ELF-filer.

    \item [\textbf{Exekveringsmotor}] En exekveringsmotor (jfr.\ eng.
    \ \emph{execution engine}) är ett program som exekverar ett programs
    instruktioner.

    \item [\textbf{Fuzzing}] Fuzzing
    är att generera indata till ett system i ett försök att hitta
    buggar eller genom frånvaron av dåligt beteende betryggas i systemets
    kvalité. Vissa verktyg genererar ny indata med genetiska algoritmer
    och vissa använder \emph{white-box} binärinstrumentation för att evaluera indata.

    \item [\textbf{Grey-box analysis}] \emph{Grey-box analysis} (jfr.\ sv.\
    \emph{grålådeanalys}) är en teknik som kombinerar \emph{black-box} och \emph{white-
        box} för att bilda en bredare analys. Ett användingsområde kan vara där
    dokumentationen är begränsad om ett programs interna struktur.

    \item [\textbf{Heap}] Heapen är ett område i minne som samtliga trådar i
    en process har tillgång till. Används för objekt som man inte vet storleken på
    innan man exekverar programmet; samt objekt som ska delas mellan en process
    trådar.

    \item [\textbf{Heuristik}] Heuristik är en praktisk metod för att lösa
    problem baserat på tidigare erfarenhet. En metod som är inte är fullständig
    utan baserad på tumregel.

    \item [\textbf{Hook}] \emph{Hook} (jfr.\ sv.\ krok) är en lista på återanropsfunktioner
    som ska köras vid ett specifikt tillstånd.

    \item [\textbf{Instrumentering}] Instrumentering (jfr.\ eng.\
    \emph{instrumentation}) är mätning av ett programs prestanda. Används för
    att hitta buggar och hitta kontrollflöden.

    \item [\textbf{IPC}] IPC (\emph{Inter-process communication}, jfr.\ sv.\
    interprocesskommunikation) är ett samlingsbegrepp för olika tekniker
    för att trådar i olika processer att kommunicera med varandra.

    \item [\textbf{KLEE}] KLEE är den symboliska exekveringsmotorn som används
    av \stoe{}.

    \item [\textbf{Maskinkod}] Maskinkod (jfr.\ eng.\ \emph{machine code}) är
    digital kod som CPU:n kan tolka och arbeta med.

    \item [\textbf{Tillståndssammanslagning}] Tillståndssammanslagning (jfr.\
    eng.\ \emph{state merging}) möjliggör att antingen automatiskt eller
    manuellt slå ihop olika stigar mellan tillstånd. Används för att öka
    presdandan.

    \item [\textbf{Systemanrop}] Systemanrop (jfr.\ eng.\ \emph{syscall}) är
    mjukvaruavbrott som program använder för att kunna anropa
    operativsystemskärnan på ett sätt som liknar funktionsanrop.

    \item [\textbf{Pekare}] Pekare (jfr.\ eng.\ \emph{pointer}) är en minnesadress som
    vanligtvis pekar på ett värde på stacken eller heapen.

    \item [\textbf{Process}] En process är en
    operativsystemabstraktion som speciellt innehåller en memory
    map(jfr.\ sv.\ minneskarta) tillsammans med ett antal trådar och
    andra operativsystemsresurser såsom fildeskriptorer (jfr.\
    eng.\ \emph{File descriptor}).

    \item [\textbf{QEMU}] QEMU (\emph{Quick Emulator}, jfr.\ sv.\ snabb
    emulator) är ett välunderhållet öppen källkods emulatorramverk
    med stöd för många plattformar och som stöder både \emph{user-space}
    emulering av en process samt emulering av ett helt system.

    \item [\textbf{Utpressningsprogram}] Utpressningsprogram (jfr.\ eng.\
    \emph{ransomware}) är en sorts skadeprogram som syftar till att göra ett it-
    system oanvändbar genom kryptering av data som endast kan avkrypteras med
    en nyckel.

    \item [\textbf{RCE}] RCE (\emph{Remote code execution}, jfr.\ sv.
    \ fjärrkodsexekvering) är en sårbarhet som tillåter körning av
    godtyckliga kommandon eller kod på en måldator.

    \item [\textbf{Reverse Engineering}] \emph{Reverse engineering} (jfr.\ sv.\
    demontering eller backlängeskonstruktion) är en process där man från en
    befintlig artefakt försöker återskapa källkodsinstruktionerna
    skrivna av de ursprungliga utvecklarna av artefakten.

    \item [\textbf{Runtime}] \emph{Runtime} (jfr.\ sv.\ körtid) är tidsspannet
    från det att ett program börjar exekveras, tills dess att programmet har
    slutat att exekveras.

    \item [\textbf{\stoe}] \stoe (\emph{The Selective Symbolic Execution
        Platform}, jfr.\ sv.\ Den selektiva exekveringsplattformen) är
    en platform som tillhandahåller symbolisk exekvering inuti den virtuella
    maskinen QEMU.\@

    \item [\textbf{SMT-lösare}] SMT-lösare (jfr.\ eng.\ \emph{Satisfiability Modulo
        Theories solver}) är ett program som kan lösa
    ekvationssystem för olika matematiska objekt. Exempel på
    teorier är modulär aritmetik eller bitvektorer. En SMT-lösare
    kan till exempel lösa ekvationer konstruerade i symbolisk
    exekvering.

    \item [\textbf{Stack}] Område i minnet som reserveras för varje
    enskild tråd. På stacken förvaras värden där man redan vid
    kompilering känner till värdets minnesstorlek.

    \item [\textbf{Starkt anslutna komponenter}] Starkt anslutna komponenter
    (jfr.\ eng.\ \emph{Strongly Connected Components}) är en riktad delgraf där
    varje nod har en väg sådant att det går att nå alla andra noder i delgrafen.

    \item [\textbf{Statisk analys}] Statisk analys (jfr.\ eng.\ \emph{static
        analysis}) är binäranalys där man endast utifrån maskinkoden på disk
    försöker dra slutsater om binärens beteende.

    % \item [\textbf{SSG}] (\emph{Sympolic State Graph},
    %       jfr.\ sv.\ symbolisk tillståndsgraf)

    \item [\textbf{Symbolisk exekvering}] Symbolisk exekvering (jfr.\ eng.\
    \emph{symbolic execution}) är att tilldela variabler symboliska, i motsats
    till konkreta värden under programmets exekvering. Med denna analysteknik
    kan enskilda körningar ge information som annars hade krävt en uttömmande
    sökning.

    \item [\textbf{Symbolisk fuzzing}] Symbolic fuzzing är en teknik som kombinerar symbolisk
    exekvering och fuzzing. Kombinationen bevarar kodstrukturen och kan
    samtidigt lösa komplexa symboliska restriktioner.

    \item [\textbf{Symbolisk operation}] Symbolisk operation (jfr.\ eng.\ \emph{symbolic
        operation}) är operationer med symboliska värden.

    \item [\textbf{Symbolisk värde}] Symbolisk värde (jfr.\ eng.\ \emph{symbolic
        value}) är en abstraktion av konkreta värden som representeras av en
    variabel. Möjliggör att analysera samtliga möjliga värden.

    \item [\textbf{Tråd}] Thread (jfr.\ eng.\ \emph{thread}) är en av flera
    parallella instruktionssekvenser inom en process.

    \item [\textbf{White-box analysis}] \emph{White-box analysis} (jfr.\ sv.
    \ vitlådsanalys) är en analysmetod som behandlar en applikations
    interna struktur i kontrast till black-box analys som enbart betraktar
    funktionaliteten.

\end{labeling}
