\begin{labeling}{begreppslista}

  \item[\textbf{Basic block}] En sekvens av instruktioner som saknar
      hopp eller branches bortsett från början och slutet av
      blocket. Är oftast expanderade till att bli så långa som möjligt

  \item [\textbf{Black-box}] Analys av ett objekt som endast betraktar dess
      yttre utseende och beteende, till skillnad från white-box-analys som även
      betraktar vad som händer inuti.

  \item [\textbf{Concolic testing}] En sorts fuzztesting där indata genereras
      genom att med en SMT-lösare lösa ekvationerna som beskriver en körning som
      ursprungligen följer körningen för en seed-indata, men vid ett angivet
      hopp avviker. Detta är en effektiv metod för att konstruera indata som tar
      speciella svår-tagna hopp.

  \item [\textbf{CTF}] CTF, eller Capture the Flag i ett
    datorsäkerhetssammanhang är en utmaning eller övning i
    att hitta gömda "flaggor" i program med avsiktliga säkerhetshål.

  \item [\textbf{ELF}] \emph{Executable and Linkable Format}. Filformatet för
      exekverbara filer på Unix.

  \item [\textbf{Fuzzing}] Att slumpmässigt generera indata till ett system i
      ett försök att hitta buggar eller genom frånvaron av dåligt beteende
      betryggas i systemets kvalité. Vissa fuzztestmotorer genererar ny indata
      med genetiska algoritmer och vissa använder white-box binärinstrumentation
      för att evaluera indata.

  \item [\textbf{QEMU}] Ett välunderhållet öppen källkods emulatorramverk med stöd
    för många plattformar och som stöder både user-space emulering av en process
    samt emulering av ett helt system.

  \item [\textbf{\stoe}] \stoe, eller \emph{The Selective Symbolic Execution Platform}, är
      en platform som tillhandahåller symbolisk exekvering inuti den virtuella
      maskinen QEMU.

  \item [\textbf{SMT Solver}] En Satisfiability modulo theories-lösare är ett
      program som kan lösa ekvationssystem för olika matematiska objekt. Exempel
      på teorier är modulär aritmetik eller bitvektorer. En SMT-lösare kan till
      exempel lösa ekvationer konstruerade i symbolisk exekvering.

  \item [\textbf{Symbolisk exekvering}] Att tilldela variabler symboliska, i
      motsats till konkreta värden under programmets exekvering. Med denna
      analysteknik kan enskilda körningar ge information som annars hade krävt
      en uttömmande sökning.

  \item [\textbf{Symbolisk fuzzing}] Symbolisk fuzzing är en teknik som
    kombinerar teknikerna symbolisk exekvering och fuzzing. Genom att kombinera
    fördelarna med båda teknikerna resulterar detta i en teknik som således
    bevarar kodstruktur och kan samtidigt lösa komplexa symboliska
    restriktioner.  
     
    
\end{labeling}
