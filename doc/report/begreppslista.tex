\begin{labeling}{begreppslista}

    \item [\textbf{Assembler}] (jfr.\ en.\ assembly) är ett
    lågnivåspråk som direkt kan översättas till maskinkod.

    \item[\textbf{Basic block}] (jfr.\ sv.\ grundblock) En sekvens av
    instruktioner som saknar hopp eller branches bortsett från
    början och slutet av blocket. Är oftast expanderade till att
    bli så långa som möjligt

    \item [\textbf{Black-box analys}] (jfr.\ sv.\ svartlådeanalys)
    Analys av ett objekt som endast betraktar dess yttre utseende
    och beteende, till skillnad från white-box-analys som även
    betraktar vad som händer inuti.

    \item [\textbf{Binär}] (jfr.\ en.\ binary) En ELF fil.

    \item [\textbf{Callbackfunktion}] (jfr.\ sv.\ ???) är en funktion
    som läggs i en hook (jfr.\ sv.\ krok) för att exekveras vid
    ett visst tillstånd.

    \item [\textbf{CFG}] (\emph{Control Flow Graph},
          jfr.\ sv.\ kontrollflödesgraf)

    \item [\textbf{Conkolisk testning}] (jfr.\ en.\ Concolic Testing)
    är en sorts fuzztesting där indata genereras genom att med en
    SMT-lösare lösa ekvationerna som beskriver en körning som
    ursprungligen följer körningen för en seed-indata, men vid ett
    angivet hopp avviker. Detta är en effektiv metod för att
    konstruera indata som tar speciella svårtagna hopp.

    \item [\textbf{CTF}] (\emph{Capture the Flag}, jfr.\ sv.\ fånga
          flaggan) är i datorsäkerhetssammanhang är en utmaning eller
    övning i att hitta gömda "flaggor" i program med avsiktliga
    säkerhetshål.

    \item [\textbf{Dwarf}] Metainformation i ELF-filer som kopplar
    maskinkod till källkod. Används vid instrumentering av
    ELF-filer.

    \item [\textbf{Dynamisk analys}] (jfr.\ en.\ dynamic analysis) är
    när en binär analyserar genom att exevera binären i en
    kontrollerad miljö för att i detalj registrera vad binären
    gör.

    \item [\textbf{ELF}] (\emph{Executable and Linkable Format},
          jfr.\ sv.\ exekverbart och länkbart format). Filformatet för
    exekverbara filer på Unix och liknande system. Innehåller
    maskinkod och länkar till andra ELF-filer.

    \item [\textbf{Exekveringsmotor}] (jfr.\ en.\ execution engine) är
    ett program som exekverar ett programs instruktioner.

    \item [\textbf{Fuzztestning}] Att slumpmässigt generera indata
    till ett system i ett försök att hitta buggar eller genom
    frånvaron av dåligt beteende betryggas i systemets
    kvalité. Vissa fuzztestmotorer genererar ny indata med
    genetiska algoritmer och vissa använder white-box
    binärinstrumentation för att evaluera indata.

    \item [\textbf{Grey-box analys}] (jfr.\ sv.\ grålådeanalys) är en
    teknik som kombinerar black-box och white-box för att bilda en
    bredare analys. Ett användingsområde kan vara där
    dokumentationen är begränsad om ett programs interna struktur.

    \item [\textbf{Grid of equals}] Designmönster inom
    interaktionsdesign för utveckling av grafiska gränssnitt där
    innehållet organiseras i geometriska former, ofta rektanglar
    som innehåller bild eller text och som placeras i lika avstånd
    till varandra och alla andra element.

    \item [\textbf{Heap}] Område i minnet som samtliga trådar i en
    process har tillgång till. Används för objekt där man inte vet
    storleken innan man exeverar programmet; samt objekt som ska
    delas mellan en process trådar.

    \item [\textbf{Heuristik}] Praktisk metod för att lösa problem
    baserat på tidigare erfarenhet.

    \item [\textbf{Hook}] (jfr.\ sv.\ krok) en lista på callback
    funktioner som ska köras vid ett specifikt tillstånd.

    \item [\textbf{Instrumentering}] (jfr.\ en.\ instrumentation)
    mätning av ett programs prestanda. Används för att hitta
    buggar och hitta kontrollflöden.

    \item [\textbf{IPC}] (\emph{Inter-process communication}, jfr. sv.
          interprocesskommunikation) är ett samlingsbegrepp på olika
    tekniker för att trådar i olika processer att kommunicera med
    varandra.

    \item [\textbf{KLEE}] Den symboliska exekveringsmotorn som används
    av \stoe{}.

    \item [\textbf{Maskinkod}] (jfr.\ en.\ machine code) är digital
    kod som CPUn kan tolka och arbeta med.

    \item [\textbf{Minne}] (jfr.\ en.\ memory, även kallat primärminne
          och RAM- minne) är dit ELF filer flyttas till när ELF filen
    ska exekveras.

    \item [\textbf{Operation}] En assemblerinstruktion med argument.

    \item [\textbf{Path Merging}]
          (jfr.\ sv.\ tillståndssammanslagning) möjliggör att antingen
    automatiskt eller manuellt slå ihop olika stigar mellan
    tillstånd.  Använd för att öka presdandan.

    \item [\textbf{Pekare}] (jfr.\ en.\ pointer) en minnesadress som
    vanligtvis pekar på ett värde på stacken eller heapen.

    \item [\textbf{Process}] En process är en
    operativsystemabstraktion som speciellt innehåller en memory
    map(jfr sv. minneskarta) tillsammans med ett antal trådar och
    andra operativsystemsresurser såsom fildeskriptorer (jfr
    en. File descriptor)

    \item [\textbf{QEMU}] (\emph{Quick Emulator}, jfr.\ sv.\ snabb
          emulator) är ett välunderhållet öppen källkods emulatorramverk
    med stöd för många plattformar och som stöder både user-space
    emulering av en process samt emulering av ett helt system.

    \item [\textbf{Ransomware}] Gisslanprogram. En sorts skadeprogram
    som syftar till att göra ett it-system oanvändbar genom
    kryptering av data som endast kan avkrypteras med en nyckel.

    \item [\textbf{Remote code execution}] En sårbarhet som tillåter
    körning av godtyckliga kommandon eller kod på en måldator.

    \item [\textbf{Reverse Engineering}] (jfr.\ sv.\ demontering eller
          backlängeskonstruktion) är en process där man från en
    befintlig artefakt försöker återskapa källkodsinstruktionerna
    skrivna av de ursprungliga utvecklarna av artefakten.

    \item [\textbf{Runtime}] (jfr.\ sv.\ körtid) är tidsspannet från
    det att ett program börjar exekveras, tills dess att
    programmet har slutat att exekveras.

    \item [\textbf{\stoe}] \emph{The Symbolic Execution Platform} är
    en platform som tillhandahåller symbolisk exekvering inuti den
    virtuella maskinen QEMU.\@

    \item [\textbf{SMT-lösare}] (\emph{Satisfiability Modulo
              Theories}, jfr.\ sv.\ ???) är ett program som kan lösa
    ekvationssystem för olika matematiska objekt.  Exempel på
    teorier är modulär aritmetik eller bitvektorer. En SMT-lösare
    kan till exempel lösa ekvationer konstruerade i symbolisk
    exekvering.

    \item [\textbf{Stack}] Område i minnet som reserveras för varje
    enskild tråd.  På stacken förvaras värden där man redan vid
    kompilering känner till värdets minnesstorlek.

    \item [\textbf{Starkt anslutna komponenter}] (jfr.\ en.\ Strongly
          Connected Components) En riktad delgraf där varje nod har en
    väg sådant att det går att nå alla andra noder i delgrafen.

    \item [\textbf{Statisk analys}] är binäranalys där man endast
    utifrån maskinkoden på disk försöker dra slutsater om binärens
    beteende.

    \item [\textbf{SSG}] (\emph{Sympolic State Graph},
          jfr.\ sv.\ symbolisk tillståndsgraf)

    \item [\textbf{Symbolisk exekvering}] Att tilldela variabler
    symboliska, i motsats till konkreta värden under programmets
    exekvering. Med denna analysteknik kan enskilda körningar ge
    information som annars hade krävt en uttömmande sökning.

    \item [\textbf{Symbolisk fuzztestning}] Teknik som kombinerar
    symbolisk exekvering och fuzzingtestning. Kombinationen
    bevarar kodstrukturen och kan samtidigt lösa komplexa
    symboliska restriktioner.

    \item [\textbf{Symbolisk operation}] (jfr.\ en.\ symbolic
          operation) operationer med symboliska värden.

    \item [\textbf{Symbolisk värde}] (jfr.\ en.\ symbolic value) är en
    abstraktion av konkreta värden som representeras av en
    variabel. Möjliggör att analysera samtliga möjliga värden.

    \item [\textbf{Tråd}] (jfr.\ en.\ thread) är en av flera
    parallella instruktionssekvenser inom en process.

    \item [\textbf{White-box analys}] (jfr.\ sv.\ vitlådsanalys) är en
    analysmetod som behandlar en applikations interna struktur i
    kontrast till black-box analys som enbart betraktar
    funktionaliteten.

\end{labeling}
