Här presenteras verktyget AMBA. Det är byggt ovanpå \stoe. AMBA är ett
verktyg för att visualisera och styra symbolisk exekvering i \stoe.
AMBA kör program med symbolisk indata och visar ett antal grafer över
exekveringen. Dessa grafer visar inte hela programmet från början,
utan uppdateras kontinuerligt under programmets exekvering.

Graferna är byggda av ett antal noder som alla representerar olika
tillstånd och kan visa vidare information om sitt tillstånd vid klick.

Den första grafen är en basic-block-graf. Här representerar varje
tillstånd i grafen ett basic block i maskinkoden. Vidare information i
den här grafen är blockets minnesadress, hur många gånger det här
blocket har tolkats (ifall det är ett självmodifierande program som
tolkas), dess motsvarande rader i källkoden ifall debugdata finns och
en disassemblering av blocket.

Den andra grafen är en komprimerad basic-block-graf. Den visar samma
graf som den första, men med linjesubgrafer sammanslagna till samma
nod. Informationen här visar spannet av sammanslagna noder.
% ^ THIS IS NOT IMPLEMENTED

Den tredje grafen visar symboliska tillstånd. Ett tillstånd här är
mellan varje branch som görs baserat på ett symboliskt
uttryck. Den vidare informationen i den här grafen är tillståndets
namn, dess sammanhörande (återanvända) id i \stoe, och uttrycket som
ledde till att den branchen togs.

Utöver att visa dessa grafer går det också att välja noder i den
symboliska tillståndsgrafen för att prioritera deras fortsatta
evaulering. Detta gör det möjligt att nå resultat i program som annars
inte nödvändigtvis skulle nå ett slut.

Grafnodernas placering i 2D hittas genom iterativ lösning för lägst
energi i ett system av attraktion-, repulsion- och externa
krafter. Utplaceringsalgoritmens parametrar kan konfigureras i realtid
vid körning av användaren.

\subsection{Begränsningar}

Alla program kan inte hanteras av AMBA. Programmen som AMBA kör måste
vara byggda för Ubuntu-22.04-x86\_64. Detta görs lättast med att
antingen bygga direkt för Ubuntu, eller för bredare support, fullt
statiskt länkad mot t.ex. musl.

Den symboliska indatan är också begränsad till att komma ifrån
stdin. Alltså kan inte programargument (argc, argv) eller filer på
disk användas.
