AMBA använder och är byggt ovanpå exekveringsmotorn \stoe{}. Två andra
exekveringsmotorer som skulle kunna ha används i detta arbete är SymQEMU och
angr. Alla dessa verktyg har beskrivits översiktligt i
avsnitt~\ref{sec:befintliga-ramverk}.

AMBA bygger inte på SymQEMU av tre anledningar. För det första har \stoe{} mer
publikt tillgänglig dokumentation än SymQEMU, vilket bedömdes underlätta
utvecklingsarbetet. För det andra genomför \stoe{}s utvecklare ett mer aktivt
underhållsarbete jämfört med SymQEMU, vilket bedömdes underlätta byggprocessen
och minska sannolikheten att stöta på problemen i den första delen av projektet.
Slutligen kan SymQEMU endast emulera enskilda program och genomför alltså inte
fullsystemsemulation. Trots att AMBA endast analyserar enskilda program ansågs
denna begränsning gynna \stoe{} genom att möjliggöra vidareutveckling till mer
generell analys. SymQEMUs stora fördel relativt \stoe{} är dess bättre
prestanda~\cite{systematic_comparison_symbex}.  Detta bedömdes inte väga upp för
dess nackdelar relativt \stoe{}.

AMBA bygger inte på angr av tre anledningar. För det första har \stoe{} bättre
prestanda än angr~\cite{systematic_comparison_symbex}. För det andra stödjer
inte heller angr fullsystemsemulation. För det tredje finns redan interaktiv
visualiserad symbolisk fuzzing byggd ovanpå angr i form av SymNav, vilket
innebär att AMBA genom att bygga på \stoe{} kan särskilja sig mer från
existerande verktyg.
