\stoe{} är ett väldigt kraftfullt verktyg som är skalbart till att användas på
storskaliga program som Google Chrome. \stoe{} tillhandahåller
kombinationen av en virtualiserad miljö — en miljö där ett program körs på en
virtuell maskin, symbolisk exekvering och dynamisk binäröversättning vilket gör
det lämpligt att användas på storskaliga program som Google Chrome ~\cite{s2e}.

Det finns konkurrenskraftiga alternativ till \stoe{}, till exempel SymQEMU men
avfärdades på grund av bristfällig publik dokumentation, i kontrast till \stoe{}
som har tillfredsställande dokumentation samt utvecklas aktivt. Personerna bakom
SymQEMU hävdar att deras verktyg är snabbare än \stoe{}
~\cite{symqemu}, trots detta valdes \stoe{}
framförallt på grund av skillnaden i dokumentationen.

Ett annat alternativ var angr men även detta avfärdades i fördel för \stoe{}
eftersom utvecklingsmiljön ansågs komplex – det var enkelt att konfigurera och börja
använda angr men till nackdel för produktiviteten. Det krävs djup förståelse av
hur angr och dess API fungerar för att kunna vara produktiv, i kontrast till
\stoe{} där användningen mellan bibliotek och applikation sker genom ett plugin.
Genom detta plugin är det enkelt att lägga till ytterligare funktionalitet i sin
applikation genom att använda funktionalitet från existerande plugin, se
avsnitt~\ref{sec:s2e} för mer om existerande plugin. 

