Beslutsanalysmodellen beskriven i \cite{foreskrifter} används för att bedöma om man behöver behöver ta hänsyn till samhälleliga eller etiska aspekter när problemställningen formuleras.
Analysen har tagit hänsyn till frågeställningarna
\begin{itemize}
    \item om vilka etiska aspekter som är relevanta för projektet;
    \item vilka nyttor eller etiska problem som kan uppkomma som ett resultat av projektets sannolika utfall; samt
    \item vilka parter som berörs av projektets utförande och projektets sannolika utfall.
\end{itemize}

En viktig etisk aspekt att ha i beaktande när man utvecklar verktyg som har som mål att underlätta sökandet efter säkerhetsbrister i programvaror är hur stor nytta verktyget är för \emph{försvarare} kontra \emph{attackerare}.
Med försvarare åsyftas de personer som har till uppgift att förhindra att ett datorsystem blir \emph{komprometterat}, medan attackerare har som mål att kompromettera datorsystem.
Att kompromettera ett datorsystem innebär att man på något sätt skadar ett datorsystems konfidentialitet; tillgänglighet; eller integritet.

Användandet av den typ av verktyg som planeras att utvecklas i projektet kommer inte att skiljas sig åt mellan försvarare och attackerare.
Medan försvararna använder verktygen för att hitta brister för att de ska veta vilka åtgärder de ska vidta för att höja säkerheten, använder attackerarna verktygen för att hitta brister som de sedan kan utnyttja för att påverka datorsystem negativt.

Personerna som berörs av projektet är främst attackerare och försvarare som får ännu ett verktyg för att genomföra dynamiska analyser av binärer.
Det anses inte vara ett problem om projektets slutprodukt är användbart, då det är vedertaget inom säkerhetssektorn att det är bättre att försvararna får bättre verktyg för att kunna utveckla nya säkerhetsmodeller; än att förhindra att attackerare får tillgång till bättre verktyg.
Om projektets slutprodukt inte anses användbart kommer projektet i realiteten inte ha någon etisk eller samhällelig påverkan.
Användarna av programvaror som analyseras av projektets slutprodukt kan påverkas om verktyget blir användbart för försvarare och attackerare, dels att fel i programvaran kan åtgärdas; men även att de kan utnyttjas av attackerare för att kompromettera användarna.
